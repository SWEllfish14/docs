\documentclass[12pt]{article}
\usepackage{graphicx} % Required for inserting images
\usepackage{hyperref}
\usepackage{makecell}
\usepackage{makeidx}
\usepackage{eurosym}
\usepackage{amsmath}
\usepackage{fancyhdr}
\usepackage{booktabs} % Per linee orizzontali migliori
\graphicspath{ {./img/} }

\newcommand{\firstPage}{
    \begin{figure}
    \centering
    \includegraphics[scale=0.5]{Swellfish_logo.png}
    \end{figure}
    \author{Andrea Veronese, Claudio Giaretta, Elena Marchioro,\\
    Davide Porporati, Francesco Naletto, Jude Vensil Barceros \\ \\
    \href{swellfish14@gmail.com}{} \\
    } 
}
\input{../templates/tabella_versioni.tex}
<<<<<<< HEAD
\newcommand{\sethdr}[1]{
		\pagestyle{fancy}
		\lhead{\includegraphics[width=1cm]{Swellfish_logo.png}}	
		\rhead{#1}
}

%\hypersetup{colorlinks=true,urlcolor=blue}

%\newcommand{\tableContent}{

	%{
		%\hypersetup{linkcolor=black}
		%\tableofcontents
	%}
=======
\newcommand{\sethdr}[1]{
		\pagestyle{fancy}
		\lhead{\includegraphics[width=1cm]{Swellfish_logo.png}}	
		\rhead{#1}
}


%\hypersetup{colorlinks=true,urlcolor=blue}

%\newcommand{\tableContent}{

	%{
		%\hypersetup{linkcolor=black}
		%\tableofcontents
	%}
>>>>>>> origin/piano_di_qualifica
%}
\begin{document}

\graphicspath{ {../templates/img/} {./img}}

\title{Piano di qualifica}

\firstPage

\sethdr{Piano di qualifica}


\maketitle

\begin{center}
	\begin{tabular}{r | l}
		\multicolumn{2}{c}{\textit{Informazioni}}                         \\
		\hline

		\textit{Redattori}    &
		[Davide Porporati, Claudio Giaretta, Francesco Naletto]\makecell{} \\

		\textit{Revisori}     &
		[Jude Vensil Braceros]\makecell{}                                 \\
		\textit{Responsabili} &
		[Andrea Veronese]\makecell{}                                      \\
		\textit{Uso}          &
		[Esterno]\makecell{}                                              \\
	\end{tabular}
\end{center}

\begin{center}
	\textbf{Descrizione}\\
	File contenente il piano di qualifica. Contiene le metriche e i criteri di accettazione dei prodotti.
\end{center}

\addversione{0.0.1}{25/04/2023}{Andrea Veronese}{Davide Porporati}{Creata struttura di base del documento}
\addversione{0.0.2}{27/04/2023}{Davide Porporati, Elena Marchioro, Francesco Naletto}{Jude Vensil Braceros}{Modificata la struttura del documento}
\addversione{0.1.0}{25/05/2023}{Claudio Giaretta, Francesco Naletto}{Francesco Naletto}{Stesura introduzione, qualità di processo e qualità di prodotto}
\addversione{0.1.1}{13/07/2023}{Claudio Giaretta}{Francesco Naletto}{Aggiornamento dei grafici e aggiunta delle ultime considerazioni del gruppo}
\addversione{1.0.0}{18/07/2023}{Andrea Veronese}{Claudio Giaretta, Davide Porporati}{Aggiornato a versione 1.0.0}
\addversione{1.0.0}{07/08/2023}{Andrea Veronese}{Claudio Giaretta, Davide Porporati}{Aggiunte metriche da usare nei test di verifica}
\makeversioni
\pagebreak

\tableofcontents
\pagebreak

\printindex



\section{Introduzione}
\subsection{Scopo del documento}
Questo documento ha lo scopo di definire le strategia di validazione e verifica addottate per garantire la qualità del prodotto.
Per raggiungere questo obbiettivo viene applicato un sistema di verifica continua sui processi e sulle attività del gruppo, questo permette di ottenere un miglioramento continuo.
Il documento non ha una funzione descrittiva, la definizione delle metriche indicate all'interno di questo documento, è presente nel documento "norme\_di\_progetto".
\section{Qualità di processo}

\subsection{Processi primari}

\subsubsection{Fornitura}
\textbf{Metriche:}
\begin{itemize}
	\item MPC01: Actual Cost (AV)
	      \begin{itemize}
		      \item \textbf{Calcolo della metrica}: Somma dei costi tracciati dal gruppo
		      \item \textbf{Valore ottimale}: $\le BAC$
		      \item \textbf{Valore accettabile}: $\le BAC$
	      \end{itemize}
	\item MPC02: Planned Value (PV)
	      \begin{itemize}
		      \item \textbf{Calcolo della metrica}: Percentuale di completamento del progetto pianificata * BAC
		      \item \textbf{Valore ottimale}: $\le BAC$
		      \item \textbf{Valore accettabile}: $\le BAC$
	      \end{itemize}
	\item MPC03: Earned Value (EV)
	      \begin{itemize}
		      \item \textbf{Calcolo della metrica}: Percentuale dell'effettivo stato di completamento del progetto * BAC
		      \item \textbf{Valore ottimale}: $\ge 0$
		      \item \textbf{Valore accettabile}: $\le BAC$
	      \end{itemize}
	\item MPC04: Cost Variance (CV)
	      \begin{itemize}
		      \item \textbf{Calcolo della metrica}:  EV - AC
		      \item \textbf{Valore ottimale}: $\ge 0\%$
		      \item \textbf{Valore accettabile}: $\ge -12\%$
	      \end{itemize}
	\item MPC05: Schedule Variance (SV)
	      \begin{itemize}
		      \item \textbf{Calcolo della metrica}:  EV - PV
		      \item \textbf{Valore ottimale}: $\ge 0\%$
		      \item \textbf{Valore accettabile}: $\ge -12\%$
	      \end{itemize}

	\item MPC06: Cost Performance Index (CPI)
	      \begin{itemize}
		      \item \textbf{Calcolo della metrica}:  EV / AC
		      \item \textbf{Valore ottimale}: $\ge 1$
		      \item \textbf{Valore accettabile}: $\ge 0,9$
	      \end{itemize}

	\item MPC07: Estimated At Completition (EAC)
	      \begin{itemize}
		      \item \textbf{Calcolo della metrica}:  BAC / CPI
		      \item \textbf{Valore ottimale}: = BAC
		      \item \textbf{Valore accettabile}: $\ge BAC - 3\% ; \le BAC + 3\% $
	      \end{itemize}
	\item MPC08: Estimate To Completition (ETC)
	      \begin{itemize}
		      \item \textbf{Calcolo della metrica}:  (BAC - EV) / CPI
		      \item \textbf{Valore ottimale}: $\ge 0\%$
		      \item \textbf{Valore accettabile}: $\le EAC$
	      \end{itemize}
\end{itemize}

\subsection{Processi di supporto}
\subsubsection{Documentazione}
\textbf{Metriche:}
\begin{itemize}
	\item MPC09: Indice di Gulpease
	      \begin{itemize}
		      \item \textbf{Calcolo della metrica}:  89 + $\frac{300*(F) - 10 * (L)}{(P)}$
		            \begin{itemize}
			            \item \textbf{L} = Numero di lettere nel testo
			            \item \textbf{P} = Numero di parole nel testo
			            \item \textbf{F} = Numero di frasi nel testo
		            \end{itemize}
		      \item \textbf{Valore ottimale}: 100 \%
		      \item \textbf{Valore accettabile}: $\ge 60\%$
	      \end{itemize}
\end{itemize}
\begin{itemize}
	\item MPC10: Errori ortografici
	      \begin{itemize}
		      \item \textbf{Calcolo della metrica}: numero errori ortografici presenti nel testo
		      \item \textbf{Valore ottimale}: 0
		      \item \textbf{Valore accettabile}: 0
	      \end{itemize}
\end{itemize}

\subsubsection{Verifica}

\textbf{Metriche:}
\begin{itemize}
\item MPC11: Statement Coverage. \\
La metrica si basa sullo statement coverage.
Indica la percentuale di statement eseguiti almeno una volta dall'insieme dei testi di unità.\\
 I valori sono forniti dalla suite di testing Jest.

\begin{center}
	\begin{tabularx}{\textwidth}{|X|X|X|}
		\hline
		\textbf{Prodotto} & \textbf{Valore accettabile } & \textbf{Valore ottimale } \\
		\hline
		Software          & $>$ 80\%                     & $>$ 95\%                     \\
		\hline
	\end{tabularx}\\[8pt]
	\mbox{}\\
\end{center}

\item MPC12: Branch Coverage \\
La metrica si basa sul branch coverage. Indica la percentuale di branch che vengono testati almeno una volta, con esito positivo. \\
Valori fonriti dal report di Jest.

\begin{center}
	\begin{tabularx}{\textwidth}{|X|X|X|}
		\hline
		\textbf{Prodotto} & \textbf{Valore accettabile } & \textbf{Valore ottimale } \\
		\hline
		Software          & $>$ 80\%                     & $>$ 95\%                     \\
		\hline
	\end{tabularx}\\[8pt]
	\mbox{}\\
\end{center}

\item MPC13: Code Coverage \\
La metrica si basa sul code coverage. Indica la percentuale di codice eseguito nella fase di testing. \\
Valori fonriti dal report di Jest.

\begin{center}
	\begin{tabularx}{\textwidth}{|X|X|X|}
		\hline
		\textbf{Prodotto} & \textbf{Valore accettabile } & \textbf{Valore ottimale } \\
		\hline
		Software          & $>$ 80\%                     & $>$ 95\%                     \\
		\hline
	\end{tabularx}\\[8pt]
	\mbox{}\\
\end{center}

\item MPC14: Condition Coverage \\
La metrica si basa sul conndition coverage. Indica la percentuale di branch che risultano almeno una volta true e almeno una volta false nell'esecuzione di un test dedicato. \\
Valori fonriti dal report di Jest.

\begin{center}
	\begin{tabularx}{\textwidth}{|X|X|X|}
		\hline
		\textbf{Prodotto} & \textbf{Valore accettabile } & \textbf{Valore ottimale } \\
		\hline
		Software          & $>$ 70\%                     & $>$ 80\%                     \\
		\hline
	\end{tabularx}\\[8pt]
	\mbox{}\\
\end{center}

\end{itemize}


\subsection{Processi organizzativi}

\section{Qualità di prodotto}
\subsection{Introduzione}
Per assicurare la qualità del prodotto, abbiamo adottato lo standard ISO/IEC 9126 come punto di riferimento. In questa sezione, forniamo i valori ottimali e accettabili per le metriche selezionate dal gruppo SWEllFish.



\subsection{Affidabilità}
\textbf{Metriche}:
\begin{itemize}
	\item MPD01: Percentuale di difetti del prodotto.
	      \begin{itemize}
		      \item Valore ottimale: 80\%.
		      \item Valore accettabile: 60\%.
		      \item Note: I valori possono essere modificati.
	      \end{itemize}
\end{itemize}


\subsection{Efficienza}
\textbf{Metriche}:
\begin{itemize}
	\item MPD02: Tempo medio di risposta.
	      \begin{itemize}
		      \item Metrica di misurazione: Secondi.
		      \item Valore ottimale: 5 secondi.
		      \item Valore accettabile: 7 secondi.
	      \end{itemize}
\end{itemize}

\subsection{Funzionalità}
\textbf{Metriche}:
\begin{itemize}
	\item MPD03: Percentuale di copertura dei requisiti.
	      \begin{itemize}
		      \item Valore ottimale: 100\% dei requisiti obbligatori e 80\% dei requisiti opzionali.
		      \item Valore accettabile: 100\% dei requisiti obbligatori.
	      \end{itemize}
\end{itemize}

\subsection{Manutenibilità}
\textbf{Metriche}:
\begin{itemize}
	\item MPD04: Percentuale di comprensibilità del codice.
	      \begin{itemize}
		      \item Valore ottimale: 85\% - 100\%.
		      \item Valore accettabile: 65\%.
	      \end{itemize}
\end{itemize}


\subsection{Portabilità}
\textbf{Metriche}:
\begin{itemize}
	\item MPD05: Percentuale di compatibilità del prodotto.
	      \begin{itemize}
		      \item Valore ottimale: 85\% - 100\%.
		      \item Valore accettabile: 60\%.
	      \end{itemize}
\end{itemize}


\subsection{Usabilità}
\textbf{Metriche:}
\begin{itemize}
	\item MPD06: Numero di errori compiuti dagli utenti durante l'utilizzo del prodotto.
	      \begin{itemize}
		      \item Valore ottimale: Inferiore a 1 errore per utente.
		      \item Valore accettabile: Inferiore a 2 errori per utente.
	      \end{itemize}
\end{itemize}


\section{Specifica di test}
\subsection{Test di accettazione}
\subsection{Test di sistema}  
  \begin{xltabular}{\linewidth}{|>{\hsize=0.6\hsize}X|>{\hsize=2.3\hsize}X|>{\hsize=0.7\hsize}X|>{\hsize=0.4\hsize}X|}						
						
	\hline						
	\textbf{ID Test} & \textbf{Descrizione} & \textbf{Requisito} & \textbf{Stato} \\						
	\hline						
	\endfirsthead						
	\hline						
	\textbf{ID Test} & \textbf{Descrizione} & \textbf{Requisito} & \textbf{Stato} \\						
	\hline						
	\endhead						
	\hline						
	\endfoot						
	TS1	 & Verificare che l'utente riesca ad effettuare correttamente l'accesso a sistema	&	RF1	&	NI	\\
	\hline						
	TS2	 & Verificare che l'utente visualizzi correttamente lo stato del sistema	&	RF2	&	NI	\\
	\hline						
	TS3	 & Verificare che l'utente sia in grado di cambiare la luminosità correttamente	&	RF3	&	NI	\\
	\hline						
	TS4	 & Verificare che il sistema visualizzi correttamente il messaggio di errore nel caso in cui l'aumento della luminosità non fosse andato a buon fine 	&	RF4	&	NI	\\
	\hline						
	TS5	& Verificare che l'utente possa visualizzare correttamente la lista delle aree illuminate	&	RF5 	&	NI	\\
	\hline						
	TS6	& Verificare che l'utente possa vedere correttamente l'elenco delle zone	&	RF6	&	NI	\\
	\hline						
	TS7	 & Verificare che l'utente possa selezionare le zone correttamente	&	RF7	&	NI	\\
	\hline						
	TS8	 & Verificare che l'utente possa diminuire la luminosità di una zona correttamente	&	RF8	&	NI	\\
	\hline						
	TS9	 & Verificare che l'utente possa accedere correttamente alla dashboard	&	RF10	&	NI	\\
	\hline						
	TS10	 & Verificare che il sistema visualizzi correttamente il messaggio di errore nel caso la diminuzione della luminosità non fosse andata a buon fine 	&	RF11	&	NI	\\
	\hline						
	TS11	 & Verificare che l'utente possa diminuire la luminosità correttamente	&	RF12	&	NI	\\
	\hline						
	TS12	 & Verificare che l'utente possa inserire una nuova area illuminata correttamente	&	RF13	&	NI	\\
	\hline						
	TS13	 & Verificare che l'utente possa rimuovere un area di illuminazione correttamente	&	RF14	&	NI	\\
	\hline						
	TS14	 & Verificare che l'utente possa accedere alla lista delle zone gestite correttamente	&	RF15	&	NI	\\
	\hline						
	TS15	 & Verificare che l'utente possa modificare le informazioni di un area illuminata	&	RF16	&	NI	\\
	\hline						
	TS16	 & Verificare che il sisteam mostri correttamente il messaggio di notifica una volta fatta la modifica all'area illuminata 	&	RF17	&	NI	\\
	\hline						
	TS17	 & Verificare che l'utente possa inserire correttamente un sensore in un area illuminata	&	RF18	&	NI	\\
	\hline						
	TS18	 & Verificare che l'utente possa accedere correttamente all'area illuminata	&	RF19	&	NI	\\
	\hline						
	TS19	 & Verificare che l'utente possa rimuovere un sensore dall'area illuminata	&	RF20	&	NI	\\
	\hline						
	TS20	 & Verificare che l'utente possa fare il logout dal sistema correttamente	&	RF21	&	NI	\\
	\hline						
	TS21	 & Verificare che l'utente possa inserire un impianto nell'elenco dei guasti correttamente	&	RF22	&	NI	\\
	\hline						
	TS22	 & Verificare che l'utente possa rimuovere un impianto dall'elenco dei guasti correttamente	&	RF23	&	NI	\\
	\hline						
	TS23	 & Verificare che l'utente possa visualizzare i dettagli di una zona correttamente	&	RF24	&	NI	\\
	\hline						
	TS24	 & Verificare che l'utente possa selezionare un lampione correttamente	&	RF25	&	NI	\\
	\hline						
	TS25	 & Verificare che l'utente possa visualizzare i dettagli di un lampione correttamente	&	RF26	&	NI	\\
	\hline						
	TS26	 & Verificare che l'utente possa inserire un nuovo lampione all'interno di un'area illuminata correttamente	&	RF27	&	NI	\\
	\hline						
	TS27	 & Verificare che l'utente possa rimuovere un lampione all'interno di un'area illuminata correttamente	&	RF28	&	NI	\\
	\hline						
	TS28	 & Verificare che l'utente possa visualizzare l'elenco delle aree illuminate con dei malfunzionamenti correttamente	&	RF29	&	NI	\\
	\hline						
	TS29	 & Verificare che l'amministratore possa poter aprire una nuova segnalazione di un guasto tramite un ticket	&	RF30	&	NI	\\
	\hline						
	TS30	 & Verificare che l'amministratire possa poter chiudere il ticket dopo aver fatto la dovuta manutenzione correttamente	&	RF31	&	NI	\\
	\hline						
	TS31	 & Verificare che il manutentore possa visualizzare i dettagli aggiuntivi di un guasto forniti dal ticket correttamente	&	RF32	&	NI	\\
	\hline						
	TS32	 & Verificare che l'utente non amministratore possa riceve le credenziali di amministratore da un superamministratore	&	RF33	&	NI	\\
	\hline						
	TS33	 & Verificare che l'utente possa consultare il manuale Lumos Minima	&	RF34	&	NI	\\
	\hline						
	TS34	 & Verificare che le nuove aree illuminate appena inserite abbiano un setup standard	&	RF35	&	NI	\\
							
	\end{xltabular}						

\subsection{Test di integrazione}
\subsection{Test di unità}
\subsubsection{Code coverage}

\section {Applicazione e valutazione delle metriche}
I grafici sono frutto di un foglio di calcolo creato dal gruppo che applica le formule per il calcolo delle metriche definite in questo documento.

\subsection{Valutazione d’insieme (Qualità di processo)}
Il lavoro è proseguito secondo le aspettative del gruppo.È stato riscontrato un calo delle ore lavorate negli sprint 8,9,10 dovuti a impegni universitari dei membri del gruppo. Questo calo ha particolarmente influenzato il grafico dello schedule variance che è effettivamente sceso sotto la soglia di tolleranza prefissata dal gruppo. Lo stesso si può riscontrare nella distanza tra il planned e l’earned value, che è cresciuta particolarmente durante quegli sprint. Il gruppo aveva comunque previsto un calo di lavoro durante gli sprint indicati, stimando di rientrare all’interno dei valori di tolleranza nei successivi sprint. 

\subsection{Planning Value, Actual Cost e Earned Value}
\begin{center}
	\includegraphics[scale=0.5]{AC_PV_EV.png}
\end{center}
\subsection{Cost Variance e Schedule Variance}
\begin{center}
	\includegraphics[scale=0.5]{Cost_Variance_Schedule_Variance.png}
\end{center}

\subsection{Eastimate at completition e Estimate to Complete}
\begin{center}
	\includegraphics[scale=0.6]{EAC_ETC.png}
\end{center}
\subsection{Cost Performance Index}
\begin{center}
	\includegraphics[scale=0.6]{CPI.png}
\end{center}
\subsection{Indice di Gulpease}
\begin{center}
	\includegraphics[scale=0.8]{Gulpease.png}
\end{center}


\end{document}
