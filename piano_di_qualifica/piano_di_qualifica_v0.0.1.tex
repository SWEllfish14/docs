\documentclass[12pt]{article}
\usepackage{graphicx} % Required for inserting images
\usepackage{hyperref}
\usepackage{makecell}
\usepackage{makeidx}
\usepackage{eurosym}
\usepackage{amsmath}
\graphicspath{ {./img/} }

\newcommand{\firstPage}{
    \begin{figure}
    \centering
    \includegraphics[scale=0.5]{Swellfish_logo.png}
    \end{figure}
    \author{Andrea Veronese, Claudio Giaretta, Elena Marchioro,\\
    Davide Porporati, Francesco Naletto, Jude Vensil Barceros \\ \\
    \href{swellfish14@gmail.com}{} \\
    } 
}
\input{../templates/tabella_versioni.tex}
\begin{document}

\graphicspath{ {../templates/img/} }

\title{Piano di progetto}

\firstPage

\maketitle

\begin{center}
    \begin{tabular}{r | l}
		\multicolumn{2}{c}{\textit{Informazioni}}\\
		\hline
		
			\textit{Redattori} &
			[Davide Porporati, Elena Marchioro, Francesco Naletto]\makecell{}\\

			\textit{Revisori} &
			[Jude Vensil Braceros]\makecell{}\\
			\textit{Responsabili} &
			[Andrea Veronese]\makecell{}\\
		      \textit{Uso} & 
                [Esterno]\makecell{}\\
    \end{tabular}
\end{center}

\begin{center}
    \textbf{Descrizione}\\
    File contenente il piano di qualifica. Contiene le metriche e i criteri di accettazione dei prodotti.
\end{center}

\pagebreak

\tableofcontents
\pagebreak

\printindex 

\addversione{0.0.0}{25/04/2023}{Andrea Veronese}{Davide Porporati}{Creata struttura di base del documento}
\addversione{0.0.1}{27/04/2023}{Davide Porporati, Elena Marchioro, Francesco Naletto}{Jude Vensil Braceros}{Creata struttura di base del documento}
\makeversioni

\section{Introduzione}
\subsection{Scopo del documento}
Questo documento ha lo scopo di definire le strategia di validazione e verifica addottate per garantire la qualità del prodotto.
Per raggiungere questo obbiettivo viene applicato un sistema di verifica continua sui processi e sulle attività del gruppo, questo permette di ottenere un miglioramento continuo.
Il documento viene redatto in maniera incrementale, ovvero viene tenuto in costante aggiornamento durante tutta la durata del progetto. 
\subsection{Scopo del capitolato}
In Italia, il prezzo del gas è aumentato significativamente, e il consumo di gas e carbone sta portando ad un aumento delle emissioni di gas serra e CO2. Per affrontare il problema del costo dell'energia, molti comuni stanno tagliando l'illuminazione pubblica, ma questo può essere evitato attraverso l'implementazione di un sistema di illuminazione pubblica ottimizzato, che consentirebbe sia la sicurezza stradale che il risparmio energetico ed economico.
Nel capitolato in questione si vuole implementare questa soluzione tramite un applicativo web. Quest'ultimo permetterebbe la gestione dell'illuminazione in modo sia automatico che manuale.
\section{Qualità di processo}
\section{Qualità di prodotto}
\section{Specifica di test}

\end{document}
