\documentclass[12pt]{article}
\usepackage{graphicx} % Required for inserting images
\usepackage{hyperref}
\usepackage{makecell}
\usepackage{makeidx}
\usepackage{eurosym}
\usepackage{fancyhdr}
\usepackage{titlesec}
\usepackage{longtable}
\graphicspath{ {./img/} }

\newcommand{\firstPage}{
    \begin{figure}
    \centering
    \includegraphics[scale=0.5]{Swellfish_logo.png}
    \end{figure}
    \author{Andrea Veronese, Claudio Giaretta, Elena Marchioro,\\
    Davide Porporati, Francesco Naletto, Jude Vensil Barceros \\ \\
    \href{swellfish14@gmail.com}{} \\
    } 
} 
\input{../templates/tabella_versioni.tex}
<<<<<<< HEAD
\newcommand{\sethdr}[1]{
		\pagestyle{fancy}
		\lhead{\includegraphics[width=1cm]{Swellfish_logo.png}}	
		\rhead{#1}
}

%\hypersetup{colorlinks=true,urlcolor=blue}

%\newcommand{\tableContent}{

	%{
		%\hypersetup{linkcolor=black}
		%\tableofcontents
	%}
=======
\newcommand{\sethdr}[1]{
		\pagestyle{fancy}
		\lhead{\includegraphics[width=1cm]{Swellfish_logo.png}}	
		\rhead{#1}
}


%\hypersetup{colorlinks=true,urlcolor=blue}

%\newcommand{\tableContent}{

	%{
		%\hypersetup{linkcolor=black}
		%\tableofcontents
	%}
>>>>>>> origin/piano_di_qualifica
%}
\graphicspath{ {../templates/img/} }

\begin{document}
\setcounter{tocdepth}{4}
\setcounter{secnumdepth}{4}
\title{Specifica Tecnica}

\firstPage
\pagebreak

\maketitle

\begin{center}
    \begin{tabular}{r | l}
		\multicolumn{2}{c}{\textit{Informazioni}}\\
		\hline
		
			\textit{Redattori} &
			[Davide Porporati, Elena Marchioro, Francesco Naletto]\makecell{}\\

			\textit{Revisori} &
			[Jude Vensil Braceros]\makecell{}\\
			\textit{Responsabili} &
			[Andrea Veronese]\makecell{}\\
		      \textit{Uso} & 
                [Esterno]\makecell{}\\
    \end{tabular}
\end{center}

\begin{center}
    \textbf{Descrizione}\\
	File contenente la specifica tecnica necessaria per la realizzazione del progetto. 
\end{center}

\pagebreak

\addversione{0.0.0}{09/08/2023}{Elena Marchioro}{Davide Porporati}{Creata struttura di base del documento}
\makeversioni

\tableofcontents

\pagebreak

\graphicspath{ {./UML/images/} }

\section{Introduzione}

\subsection{Scopo del documento}
Nel seguente documento vengono illustrate e motivate le scelte architetturali decise. Vengono riportati i diagrammi delle classi per l'architettura e le funzionalità principali,
il diagramma ER della base di dati e infine una sezione dalla quale si può verificare lo stato di avanzamento del prodotto grazie a una tabella che illustra
i requisiti soddisfatti.

\subsection{Scopo del prodotto}
L'obiettivo richiesto dall'azienda proponente è la realizzazione di una WebApp che permetta a degli utenti registrati di gestire l'impianto di illuminazione
di una zona in modo manuale e automatico. Nel documento viene riportata l'architettura del sistema per i vari servizi e i design pattern utilizzati.

\subsection{Riferimenti}
\subsubsection{Riferimenti normativi}
\begin{itemize}
	\item Norme di progetto
	\item Capitolato d'appalto C2 - \href{https://www.math.unipd.it/~tullio/IS-1/2022/Progetto/C2.pdf}{Lumos Minima}
\end{itemize}
\subsubsection{Riferimenti informativi}
\begin{itemize}
	\item Analisi dei requisiti
	\item Slide P2 del corso di ingegneria del software - \href{https://www.math.unipd.it/~rcardin/swea/2023/Diagrammi%20delle%20Classi.pdf}{Diagrammi delle classi}
\end{itemize}

\section{Architettura del prodotto}
\subsection{Descrizione generale}
\subsection{Diagramma delle classi}
\subsection{Design Pattern}
\section{Base di Dati}
\subsection{Progettazione concettuale}
\subsection{Progettazione logica}
\section{Requisiti soddisfatti}
\subsection{Tabella requisiti soddisfatti}

\begin{xltabular}{\linewidth}{|>{\hsize=0.6\hsize}X|>{\hsize=1.8\hsize}X|>{\hsize=1\hsize}X|>{\hsize=0.6\hsize}X|}
	% \begin{xltabular}{200pt}{|c|X|c|X|}
	\hline
	\textbf{Requisito} & \textbf{Descrizione} & \textbf{Classificazione} & \textbf{Stato} \\
	\hline
	\endfirsthead
	\hline
	\textbf{Requisito} & \textbf{Descrizione} & \textbf{Classificazione} & \textbf{Stato} \\
	\hline
	\endhead
	\hline
	\endfoot
	RF1	 & L'utente deve poter fare il login al sistema	 & Obbligatorio	 & Non soddisfatto \\
	\hline				
	RF2	 & L'utente visualizza lo stato del sistema	 & Obbligatorio	 & Non soddisfatto \\
	\hline				
	RF3	 & L'utente deve poter aumentare la luminosità di una zona	 & Obbligatorio	 & Non soddisfatto \\
	\hline				
	RF4	 & Il sistema deve visualizzare un messaggio d'errore se non si è potuto aumentare la luminosità	 & Obbligatorio	 & Non soddisfatto \\
	\hline	
	RF5 & L'utente deve poter vedere l'elenco delle aree illuminate	 & Obbligatorio	 & Non soddisfatto \\
	\hline
	RF6 & L'utente deve poter vedere l'elenco delle zone & Obbligatorio	 & Non soddisfatto \\
	\hline
	RF7	 & L'utente deve poter selezionare le zone su cui operare	 & Obbligatorio	 & Non soddisfatto  \\
	\hline	
	RF8	 & L'utente deve poter diminuire la luminosità di una zona	 & Obbligatorio	 & Non soddisfatto \\
	\hline				
	RF10	 & L'utente deve poter accedere alla dashboard	 & Obbligatorio	 & Non soddisfatto \\
	\hline										
	RF11	 & Il sistema deve visualizzare un messaggio d'errore nel caso l'operazione di diminuzione della luminosità non fosse andata a buon fine	 & Obbligatorio	 & Non soddisfatto \\
	\hline				
	RF12	 & L'utente deve poter diminuire la luminosità	 & Obbligatorio	 & Non soddisfatto \\
	\hline				
	RF13	 & L'utente deve poter inserire una nuova area illuminata	 & Obbligatorio	 & Non soddisfatto \\
	\hline				
	RF14	 & L'utente deve poter rimuovere un area illuminata	 & Obbligatorio	 & Non soddisfatto \\
	\hline				
	RF15	 & L'utente deve poter accedere alla lista delle zone gestite	 & Obbligatorio	 & Non soddisfatto \\
	\hline				
	RF16	 & L'utente deve poter modificare le informazioni di un'area illuminata	 & Obbligatorio	 & Non soddisfatto \\
	\hline				
	RF17	 & Il sistema mostra un messaggio di notifica una volta effettuata la modifica ad un area illuminata & Obbligatorio	 & Non soddisfatto \\
	\hline				
	RF18	 & L'utente deve poter inserire un nuovo sensore in una area illuminata	 & Obbligatorio	 & Non soddisfatto \\
	\hline				
	RF19	 & L'utente deve poter accedere all'area illuminata	 & Obbligatorio	 & Non soddisfatto \\
	\hline 				
	RF20	 & L'utente deve poter rimuovere un sensore da una zona illuminata	 & Obbligatorio	 & Non soddisfatto \\
	\hline				
	RF21	 & L'utente deve poter fare il logout dal sistema	 & Obbligatorio	 & Non soddisfatto \\
	\hline										
	RF22	 & L'utente deve poter inserire un impianto nell'elenco dei guasti	 & Obbligatorio	 & Non soddisfatto \\
	\hline				
	RF23	 & L'utente deve poter rimuovere un impianto dall'elenco dei guasti	 & Obbligatorio	 & Non soddisfatto \\
	\hline				
	RF24	 & L'utente deve poter visualizzare i dettagli di una zona	 & Obbligatorio	 & Non soddisfatto \\
	\hline				
	RF25	 & L'utente deve poter selezionare un lampione	 & Obbligatorio	 & Non soddisfatto\\
	\hline				
	RF26	 & L'utente deve poter visualizzare i dettagli di un lampione	 & Obbligatorio	 & Non soddisfatto\\
	\hline				
	RF27	 & L'utente deve poter inserire un nuovo lampione all'interno di un'area illuminata	 & Obbligatorio	 & Non soddisfatto \\
	\hline				
	RF28	 & L'utente deve poter rimuovere un lampione all'interno di un'area illuminata	 & Obbligatorio	 & Non soddisfatto \\
	\hline				
	RF29	 & L'utente deve poter visualizzare l'elenco delle aree illuminate con dei malfunzionamenti	 & Obbligatorio	 & Non soddisfatto \\
	\hline				
	RF30	 & L'amministratore deve poter aprire una nuova segnalazione di un guasto tramite un ticket	 & Obbligatorio	 & Non soddisfatto \\
	\hline				
	RF31	 & L'amministratore deve poter chiudere il ticket dopo aver fatto la dovuta manutenzione & Obbligatorio	 & Non soddisfatto \\
	\hline				
	RF32	 & Il manutentore deve poter visualizzare i dettagli aggiuntivi di un guasto forniti dal ticket	 & Obbligatorio	 & Non soddisfatto \\
	\hline				
	RF33	 & L'utente non amministratore riceve le credenziali da amministratore da un superamministratore & Desiderabile	 & Non soddisfatto \\
	\hline				
	RF34	 & L'utente consulta il manuale Lumos Minima	 & Desiderabile	 & Non soddisfatto \\
	\hline				
	RF35	 & Le nuove aree illuminate appena inserite hanno un setup standard	 & Desiderabile	 & Non soddisfatto \\
	\hline				
	\end{xltabular}
	
	
	\subsection{Qualità}
	\begin{tabular}{ |p{1.8cm}|p{5.2cm}|p{3cm}| p{2cm}| }
	\hline
	Requisito& Descrizione &Classificazione & Stato \\
	\hline
	RQ1 & La webapp deve essere sviluppata seguendo le regole descritte nel documento Norme di progetto & Obbligatorio & Non soddisfatto \\
	RQ2 & Devono essere sviluppati dei test con una copertura minima dell'80\% e correlati di report & Obbligatorio & Non soddisfatto\\
	RQ3 & Deve essere prodotto un documento sulle scelte implementative e progettuali & Obbligatorio & Non soddisfatto \\
	RQ4 & Deve essere prodotto un documento sui problemi aperti e sulle eventuali soluzioni da esplorare & Obbligatorio & Non soddisfatto \\
	RQ5 & Fornire un’analisi rispetto al carico massimo supportato in numero di dispositivi e di quale sarebbe il servizio cloud più adatto per supportarlo analizzando prezzo, stabilità del servizio ed assistenza.  &  Facoltativo & Non soddisfatto \\
	\hline
	
	\end{tabular}
\subsection{Dati soddisfazione requisiti}
percentuali varie

\end{document}