\documentclass[12pt]{article}
\usepackage{graphicx} % Required for inserting images
\usepackage{makeidx}
\title{Valutazione Capitolati}
\makeindex
\begin{document}
\begin{figure}
\centering
\includegraphics[scale=0.5]{Swellfish_logo.png}
\end{figure}
\author{Andrea Veronese, Claudio Giaretta, Elena Marchioro,\\
Davide Porporati, Francesco Naletto, Jude Vensil Barceros \\ \\
 \href{swellfish14@gmail.com}{} \\
} 
\date{Marzo 2023}

\maketitle

\begin{center}
    \begin{tabular}{r | l}
		\multicolumn{2}{c}{\textit{Informazioni}}\\
		\hline
		
			\textit{Redattori} &
			\makecell[Andrea Veronese]{\redattori}\\
		
			\textit{Revisori} &
			\makecell[l]{\revisori}\\
			\textit{Responsabili} &
			\makecell[l]{\responsabili}\\
		      \textit{Uso} & 
                \makecell[l]{\uso}\\
\end{tabular}
\end{center}


\tableofcontents
\printindex 
\section{Valutazione capitolati}
Il secondo lotto di Ingegneria del Software è stato caratterizzato da una ridotta disponibilità di capitolati, pertanto tutti i progetti sono stati valutati attentamente, studiando per intero le specifiche richieste e le modalità di svolgimento.
Dato l'interesse del gruppo per due capitolati specifici, Lumos Minima e Trustify, con le rispettive aziende è stato richiesto un colloquio per discutere apertamente le specifiche e valutare le tecnologie da utilizzare.
Per il restante capitolato, Personal Identity Wallet, il team ha concordato di non fissare un incontro con il proponente, ma ha comunque valutato per intero la proposta.

\section{Capitolato Scelto}
Dopo aver fissato un incontro con l'azienda porponente, Imola Informatica, il gruppo è rimasto colpito dalla libertà di scelta nelle teconologie impiegabili, pertanto ha deciso di candidarsi per questo capitolato.
I motivi specifici che hanno spinto i componenti del gruppo a scegliere questo capitolato a scapito di Trustify sono di seguito riportati:
\begin{itemize}
    \item Parte dei componenti del gruppo hanno dimostrato interesse per la parte riguardante l'interfacciamento con la componente hardware del progetto
    \item Disponibilità da parte del proponente ad inviare il materiale necessario per il progetto
    \item L'utilità effettiva di questo progetto è stato uno dei motivi principali che ha portato il team a sceglierlo.
    \item L'azienda proponente di è dimostrata disponibile nel fornire supporto in caso di problemi e formazione per quanto possibile nell'uso delle tecnologie nuove con cui ci interfacceremo
     \item Abbiamo ritenuto l'integrazione con sistemi hardware(anche se non disponibili attualmente) di particolare interesse e un valore aggiunto rispetto gli altri capitolati.
\end{itemize}
\section{Analisi dei documenti di specifica}
\subsection{Capitolato 2 - Lumos Minima (Proponente: Imola Informatica s.p.a.)}
    Il progetto richiede la realizzazione di un applicazione web che permetta di gestire da remoto un sistema di illuminazione tramite l'intervento di operatori specializzati. L' hardware menzionato nel capitolato non è disponibile per questioni di scarsità di materiale, e viene sostituito da un simulatore python controllato tramite API rest.
    Il documento di specifica chiede di poter gestire l'illuminazione in maniera automatica e manuale, permettendo di localizzare eventuali guasti e di intervenire tempestivamente per risolverli tramite l'intervento di operatori specializzati.
    Le tecnologie richieste per questo capitolato sono a scelta libera, lasciando agli sviluppatori il massimo grado di libertà, purchè si implementino tecnologie open-source che non richiedano il pagamento di canoni "mensili" agli utilizzatori di tale sistema.    
     
\subsection{Capitolato 3 - Digital Wallet (Proponente: InfoCert s.p.a.)}
    Il capitolato propone di creare un'identita virtuale utilizzabile in tutta l'Unione Europea, valida anche per fini legali. Lo scopo del progetto è quello di unire la comodità di un Single Sign On, come quello fornito da UNIPD con la necessità che tale identità sia verificata, servizio offerto da SPID.
    Personal Identity Wallet si propone come soluzione semplice che rappresenti il perfetto connubio fra le due necessità presentate in precedenza.
    Il proponente richiede che la struttura del Wallet sia fruibile tramite formati ben noti, come JSON e richiede l'utilizzo del protocollo OpenID4VC per la verifica, oltre a elevati standard di sicurezza che impediscano il furto di identità.
    Il gruppo ha valutato a fondo il capitolato, ma non vi è stato particolare interesse per il progetto, pertanto è stato scartato.

\subsection{Capitolato 7 - Trustify (Proponente: Sync Lab S.r.l.)}
    Il proponente richiede lo sviluppo di una web app che permetta di rilasciare delle recensioni verificate mediante smart contract. Il progetto nasce dalla necessità di avere recensioni veritiere e legate ad una transazione realmente avvenuta, e il suo scopo è quello di contrastare recensioni fasulle e di evitare la censura da parte dell'azienda venditrice.
    Per realizzare quanto richiesto, è necessario implementare una blockchain,i smart contract e effettuare le operazioni tramite api rest.
    Il gruppo ha concordato un colloquio con Synclab dove sono stati discussi i dettagli implementativi e le teconologie impiegabili.
    Il gruppo ha preferito un altro capitolato rispetto a questo, in quanto non tutti i componenti trovano la tecnologia blockchain interessante e alcuni la consideravano ancora acerba e non largamente impiegabile.

\end{document}

