\documentclass[12pt]{article}
\usepackage{graphicx} % Required for inserting images
\usepackage{hyperref}
\usepackage{makecell}
\usepackage{makeidx}
\usepackage{eurosym}
\graphicspath{ {./img/} }

\newcommand{\firstPage}{
    \begin{figure}
    \centering
    \includegraphics[scale=0.5]{Swellfish_logo.png}
    \end{figure}
    \author{Andrea Veronese, Claudio Giaretta, Elena Marchioro,\\
    Davide Porporati, Francesco Naletto, Jude Vensil Barceros \\ \\
    \href{swellfish14@gmail.com}{} \\
    } 
}
\input{../templates/tabella_versioni.tex}
\begin{document}

\graphicspath{ {../templates/img/} }

\title{Glossario}

\firstPage

\maketitle

\begin{center}
    \begin{tabular}{r | l}
		\multicolumn{2}{c}{\textit{Informazioni}}\\
		\hline
		
			\textit{Redattori} &
			[Davide Porporati, Elena Marchioro, Francesco Naletto]\makecell{}\\

			\textit{Revisori} &
			[Jude Vensil Barceros]\makecell{}\\
			\textit{Responsabili} &
			[Andrea Veronese]\makecell{}\\
		      \textit{Uso} & 
                [Esterno]\makecell{}\\
    \end{tabular}
\end{center}

\begin{center}
    \textbf{Descrizione}\\
    File contenente un glossario dei termini fino ad ora utilizzati per il progetto.
\end{center}

\pagebreak

\tableofcontents
\pagebreak

\printindex 

\addversione{0.0.0}{27/04/2023}{Davide Porporati, Elena Marchioro, Francesco Naletto}{Jude Vensil Barceros}{Stesura della base del documento }
\makeversioni

\section{Glossario}
\begin{itemize}
    \item API: ono un insieme di definizioni e protocolli con i quali vengono realizzati e integrati software applicativi. Possono essere considerate come un contratto tra un fornitore di informazioni e l'utente destinatario di tali dati: l'API stabilisce il contenuto richiesto dal consumatore (la chiamata) e il contenuto richiesto dal produttore (la risposta).
    \item Area illuminata: rappresenta un sottoinsieme degli impianti di illuminazione che vengono gestiti in modo uniforme per intensità luminosa
    \item Diagramma di Gantt: Diagramma che rappresenta un'arco temporale e i gradi di completamento delle task aggiunte in questo dato intervallo di tempo.
    \item Intensità luminosa: rappresenta un valore da 0 (impianto spento) a 10 (illuminazione massima) configurabile in ogni singolo impianto di illuminazione
    \item Interfaccia: dispositivo di collegamento con cui un software assicura la comunicazione tra due sistemi altrimenti incompatibili, oppure tra unità centrali e periferiche.
    \item MQTT: è un protocollo di messaggistica basato su standard, utilizzato per la comunicazione tra macchine. I sensori intelligenti, i dispositivi indossabili e altri dispositivi IoT devono in genere trasmettere e ricevere dati su una rete con risorse limitate e larghezza di banda limitata. Questi dispositivi IoT utilizzano MQTT per la trasmissione dei dati, in quanto è facile da implementare e può comunicare i dati IoT in modo efficiente.
    \item Piano di Progetto (PdP): calendario di massima di un progetto riportante la stima dei costi di realizzazione, dei rischi attesi e della loro mitigazione e della suddivisione del lavoro in molteplici periodi successivi. 
    \item Piano di Qualifica(PdQ): specifica gli obiettivi quantitativi di qualità di prodotto e di processo, oltre a misurare il raggiungimento di tali obiettivi allo stato corrente pone le fondamenta per retrospettive e iniziative di auto-miglioramento
    \item Processo: insieme di attività correlate che trasformano bisogni in prodotti, consumando risorse durante l'esecuzione.
    \item Rilevamento della presenza: rilevamento della presenza di persone in un area illuminata da uno dei sensori del sistema
    \item Web App: applicazione la cui fruizione avviene tramite browser
    \item Sistema: sistema informatico che esegue azioni su impianti luminosi in maniera automatica ed in 
    tempo reale, prendendo decisioni in base alle configurazioni inserite ed ai dati rilevati dai sensori
    \item Way of Working: letteralmente "modo di lavorare". E' la base di cui si dota il gruppo per lavorare coerentemente e in maniera professionale
\end{itemize}
\end{document}
