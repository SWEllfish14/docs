\documentclass[12pt]{article}
\usepackage{graphicx} % Required for inserting images
\graphicspath{ {images/} }
\begin{document}
\begin{header}
\title{SwellFish}
\begin{figure}
\centering
\includegraphics[scale=0.5]{Swellfish_logo_png.png}
\end{figure}
\author{Andrea Veronese, Claudio Giaretta, Elena Marchioro,\\
Davide Porporati, Francesco Naletto, Jude Vensil Barceros \\ \\
 \href{swellfish14@gmail.com}{} \\
} 
\date{20 Aprile 2023}
\end{header}


\maketitle
\begin{center}
    \begin{tabular}{r | l}
		\multicolumn{2}{c}{\textit{Informazioni}}\\
		\hline
			\textit{Redattori} &
			\makecell[Elena Marchioro]{\redattori}\\
			\textit{Revisori} &
			\makecell[l]{\revisori}\\
			\textit{Responsabili} &
			\makecell[l]{\responsabili}\\
		      \textit{Uso} & 
                \makecell[Interno]{\uso}\\
\end{tabular}
\end{center}


\tableofcontents
\printindex 
\section{Riunione del 20/04/2023 - Bozza Way of Working}
Durante la riunione sono stati discussi i seguenti argomenti:
\begin{itemize}
    \item decisione ordine di rotazione dei ruoli secondo l'ordine alfabetico cambiandoli ogni settimana.
    \item organizzazione dei documenti definizione della struttura, versionamento, verifica e validazione.
    \item organizzazione delle modalità di comunicazione ed è stata fissata una riunione ogni venerdì tramite Discord in cui verranno decisi gli obiettivi della settimana e discusse le cose fatte durante quella appena trascorsa.
    \item gli sprint si è deciso avranno durata di una settimana
    \item il responsabile ogni settimana si occuperà di aggiornare la board del progetto su github e assegnerà a ognuno il suo compito.
\end{itemize}
La riunione è durata 90 minuti ed è stata prodotta un prima bozza del way of working.
\end{document}
