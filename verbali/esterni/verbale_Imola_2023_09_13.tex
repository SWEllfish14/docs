\documentclass[12pt]{article}
\usepackage{graphicx} % Required for inserting images
\usepackage{hyperref}
\usepackage{makecell}
\usepackage{makeidx}

\begin{document}
\graphicspath{ {../../templates/img} }
\graphicspath{ {./img/} }

\newcommand{\firstPage}{
    \begin{figure}
    \centering
    \includegraphics[scale=0.5]{Swellfish_logo.png}
    \end{figure}
    \author{Andrea Veronese, Claudio Giaretta, Elena Marchioro,\\
    Davide Porporati, Francesco Naletto, Jude Vensil Barceros \\ \\
    \href{swellfish14@gmail.com}{} \\
    } 
}
\title{Verbale incontro con Imola 2023-09-13}
\firstPage
\maketitle

\begin{center}
    \begin{itemize}
        \item[] Data: 
        \item[] Ora inizio: 17:00
        \item[] Ora fine: 17:35
        \item[] Partecipanti: Andrea Veronese, Claudio Giaretta, Davide Porporati, Elena Marchioro, Francesco Naletto, Jude Vensil Braceros
        \item[] Partecipanti esterni: Lorenzo Patera
        \end{itemize}
    \begin{tabular}{r | l}
		\multicolumn{2}{c}{\textit{Informazioni}}\\
		\hline
		
			\textit{Redattori} &
			[Claudio Giaretta]\makecell{}\\
		
			\textit{Revisori} &
			[Davide Porporati]\makecell{}\\
			\textit{Responsabili} &
			[Jude Vensil Braceros]\makecell{}\\
		      \textit{Uso} & 
                [Interno]\makecell{}\\
\end{tabular}
\end{center}

\tableofcontents
\printindex
\section{Incontro Imola Informatica 2023/09/13}
\subsection{Scopo della riunione}
\begin{itemize}
    \item Mostare il prodotto finito per la revisione della Product Baseline
\end{itemize}
\subsection{Svolgimento della riunione}
\subsubsection{Presentazione lavoro svolto}
È stato mostrato al proponente tutto ciò che è stato svolto fino ad oggi.
Abbiamo quindi presentato il prodotto finito, mostrando tutti i particolari dell'applicazione.
È stato verificato se soddisfa tutti i requisiti obbligatori e se soddisfa anche dei requisiti facoltativi o desiderabili.
\\
\subsubsection{Domande riguardati login}
\begin{itemize}
    \item È stato chiesto se si poteva accedere al login direttamente da URL e se l'aggiunta e la modifica erano thread safe
    \subitem Risposta: Non è possibile bypassare la parte di login e aggiunta e modifica sono di tipo thread safe
\end{itemize}
\\
\subsubsection{Domande riguardati parte test}
\begin{itemize}
    \item È stato chiesto se è stata verificata la parte di testing e che percentuali soddisfa il prodotto
    \subitem Risposta: La copertura di test dev'essere dell'80/100 e il nostro prodotto soddisfa tutti i test svolti, alle volte anche col massimo della parcentuale
\end{itemize}
\\
\subsubsection{Domande riguardati sistema di ticketing}
Dal gruppo è stato inoltre domandato: 
\begin{itemize}
    \item Qualche consiglio per implementare il sistema di ticketing
    \subitem Risposta: Il sistema di ticketing è un requisito facoltativo e non obbligatorio ma se vogliamo implementarlo è consigliato utilizzare Jira
\end{itemize}
\\
Il proponente si è comunque detto molto soddisfatto del lavoro svolto. 
\end{document}