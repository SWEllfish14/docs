\documentclass[12pt]{article}
\usepackage{graphicx} % Required for inserting images
\usepackage{hyperref}
\usepackage{makecell}
\usepackage{makeidx}

\begin{document}
\graphicspath{ {../../templates/img} }
\graphicspath{ {./img/} }

\newcommand{\firstPage}{
    \begin{figure}
    \centering
    \includegraphics[scale=0.5]{Swellfish_logo.png}
    \end{figure}
    \author{Andrea Veronese, Claudio Giaretta, Elena Marchioro,\\
    Davide Porporati, Francesco Naletto, Jude Vensil Barceros \\ \\
    \href{swellfish14@gmail.com}{} \\
    } 
}
\title{Verbale incontro con Imola 21-04-2023}
\firstPage
\maketitle

\begin{center}
    \begin{itemize}
        \item[] Data: 21/04/2023
        \item[] Ora: 15:00
        \item[] Durata: 00.35
        \item[] Partecipanti: Andrea Veronese, Claudio Giaretta, Davide Porporati, Elena Marchioro, Francesco Naletto, Jude Vensil Braceros
        \item[] Partecipanti esterni: Lorenzo Patera
        \item[] 
        \end{itemize}
    \begin{tabular}{r | l}
		\multicolumn{2}{c}{\textit{Informazioni}}\\
		\hline
		
			\textit{Redattori} &
			[Elena Marchioro]\makecell{}\\
		
			\textit{Revisori} &
			[Jude Vensil Braceros]\makecell{}\\
			\textit{Responsabili} &
			[Andrea Veronese]\makecell{}\\
		      \textit{Uso} & 
                [Interno]\makecell{}\\
\end{tabular}
\end{center}

\tableofcontents
\printindex
\section{Incontro Imola Informatica 21/04/2023}
\subsection{Riassunto}
Sono stati discussi i seguenti punti relativi al progetto:
\begin{itemize}
    \item i punti necessari (in bold nella presentazione del capitolato) e quelli facoltativi (sottolineati) da soddisfare
    \item framework consigliati per la realizzazione sono Angular, React e Javascript
    \item sono da simulare circa 20-30 lampioni da poi dividere per aree
    \item illuminazione da controllare con API Rest o Mosquitto
    \item gestione della sessione
    \item login per operatori
    \item form di registrazione per approvare nuovi operatori e dar loro username e password per loggarsi (facoltativo)
    \item necessario server con python3 per il simulatore (può fornirlo l'azienda)
    \item consigliato l'uso di API Tester
    \item è stato spiegato l'utilizzo di Mosquitto e API Rest
    \item ci è stato fornito il simulatore
    \item da creare un simulatore per il sensore
    \item ogni area può essere fornita di uno o più sensori, ma non uno per lampada
    \item va simulato un ambiente reale
\end{itemize}
\subsection{Riunioni future}
Si è deciso di fare subito una riunione per parlare dei punti esposti.
\end{document}
