\documentclass[12pt]{article}
\usepackage{graphicx} % Required for inserting images
\usepackage{hyperref}
\usepackage{makecell}
\usepackage{makeidx}

\begin{document}
\graphicspath{ {../../templates/img} }
\graphicspath{ {./img/} }

\newcommand{\firstPage}{
    \begin{figure}
    \centering
    \includegraphics[scale=0.5]{Swellfish_logo.png}
    \end{figure}
    \author{Andrea Veronese, Claudio Giaretta, Elena Marchioro,\\
    Davide Porporati, Francesco Naletto, Jude Vensil Barceros \\ \\
    \href{swellfish14@gmail.com}{} \\
    } 
}
\title{verbale 14-03-2023}

\firstPage
\maketitle

\begin{center}
    \begin{tabular}{r | l}
		\multicolumn{2}{c}{\textit{Informazioni}}\\
		\hline
		
			\textit{Redattori} &
			[Andrea Veronese]\makecell{}\\
		
			\textit{Revisori} &
			[Claudio Giaretta]\makecell{}\\
			\textit{Responsabili} &
			[Elena Marchioro]\makecell{}\\
		      \textit{Uso} & 
                [Esterno]\makecell{}\\
\end{tabular}
\end{center}


\tableofcontents
\printindex 
\section{Riunione del 14/03/2023 - Incontro con Imola Informatica}
In questa data il gruppo ha fissato un incontro con il proponente Imola Informatica per il progetto Lumos Minima.
In questa riunione sono stati chiariti dettagli tecnici non riportati nel capitolato e sono state fatte delle domande da parte dei componenti del gruppo su vari aspetti del progetto.
In particolare abbiamo discusso i seguenti punti:
\begin{itemize}
    \item teconologie da utilizzare: come da capitolato non ci sono vincoli particolari su cosa utilizzare per la realizzazione. E' stato ribadito che la pagina web deve essere responsive e di non usare framework poco conosciuti per non complicare il processo di revisione e di preferire standard open-source che non necessitino pagamenti di licenze. E' stato inoltre suggerito un sistema di ticketing utilizzabile dagli operatori per aprire una segnalazione di guasto.
    \item ruoli degli utilizzatori: abbiamo concordato la presenza di un solo amministratore di sistema che può gestire per intero le zone di illuminazione e i controlli sulla luminosità, mentre gli altri utilizzatori avranno ruoli con funzionalità limitate.
    \item implementazione hardware: come per i gruppi del primo lotto anche in questo caso è stata ribadita la difficoltà nel reperire sensori di luminosità e raspberry pi necessari per comandare fisicamente la lampadina, pertanto questa parte verrà emulata con un simulatore python tramite api rest.
    \item testing: Imola Informatica mette a disposizione una macchina virtuale accessibile tramite VPN
    \item Richiesta bonus: se avanza tempo al termine dello sviluppo delle funzionalità preventivate, l'azienda ha richiesto di implementare anche un sistema di registrazione al sistema per i nuovi operatori addetti alla manutenzione e/o gestione del sistema.
\end{itemize}
La riunione è durata poco meno di un ora e oltre ad aver riguardato per intero il documento di specifica, del tempo è stato dedicato ai punti riportati in precedenza, risolvendo ogni dubbio sorto sino ad ora
\end{document}
