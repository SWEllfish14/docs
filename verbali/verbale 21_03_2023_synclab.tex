\documentclass[12pt]{article}
\usepackage{graphicx} % Required for inserting images
\graphicspath{ {images/} }
\begin{document}
\begin{header}
\title{SwellFish}
\begin{figure}
\centering
\includegraphics[scale=0.5]{Swellfish_logo_png.png}
\end{figure}
\author{Andrea Veronese, Claudio Giaretta, Elena Marchioro,\\
Davide Porporati, Francesco Naletto, Jude Vensil Barceros \\ \\
 \href{swellfish14@gmail.com}{} \\
} 
\date{Marzo 2023}
\end{header}


\maketitle
\begin{center}
    \begin{tabular}{r | l}
		\multicolumn{2}{c}{\textit{Informazioni}}\\
		\hline
		
			\textit{Redattori} &
			\makecell[Andrea Veronese]{\redattori}\\
		
			\textit{Revisori} &
			\makecell[l]{\revisori}\\
			\textit{Responsabili} &
			\makecell[l]{\responsabili}\\
		      \textit{Uso} & 
                \makecell[l]{\uso}\\
\end{tabular}
\end{center}


\tableofcontents
\printindex 
\section{Riunione del 21/03/2023 - Incontro con SyncLab}
In questa data il gruppo ha concordato un incontro con l'azienda SyncLab per discutere meglio alcuni aspetti del capitolato riguardante il progetto Trustify.
In questa riunione l'azienda ha presentato per intero il capitolato, spiegando nel dettaglio le necessità e le tecnologie da utilizzare per realizzare l'applicativo.
In particolare abbiamo disusso le seguenti tematiche:
\begin{itemize}
    \item Uso delle blockchain: il proponente ci ha spiegato perchè l'uso di questa tecnologia è vincolante per il progetto. E' l'unica soluzione in grado di fornire uno smart contract generico che viene implementato per ogni transazione.
    \item Necessità che il progetto sia agnostico: è richiesto che il progetto sia applicabile a qualsiasi tipo di azeinda che svolge operazioni di e-commerce.
    \item Immutabilit\'{a} di una recensione inserita a sistema: quando una recensione è inserita e di conseguenza verificata, il cliente deve poter avere la possibilità di modificarla, mentre l'azienda venditrice non può in alcun modo cancellarla, al massimo può limitarsi a nasconderla.
\end{itemize}
Questo incontro è durato circa 45 minuti, durante i quali sono stati chiariti i punti precedenti e le motivazioni specifiche che hanno portato il proponente ad inserire queste richieste. Oltre a ciò è stata offerta dall'azienda una panoramica di sintesi sui requisiti funzionali del progetto.
\end{document}
