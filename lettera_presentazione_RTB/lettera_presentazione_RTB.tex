\documentclass[12pt]{article}
\usepackage{graphicx} % Required for inserting images
\usepackage{hyperref}
\usepackage{makecell}
\usepackage{makeidx}
\usepackage{eurosym}
\usepackage{fancyhdr}
\usepackage{titlesec}
\graphicspath{ {./img/} }

\newcommand{\firstPage}{
    \begin{figure}
    \centering
    \includegraphics[scale=0.5]{Swellfish_logo.png}
    \end{figure}
    \author{Andrea Veronese, Claudio Giaretta, Elena Marchioro,\\
    Davide Porporati, Francesco Naletto, Jude Vensil Barceros \\ \\
    \href{swellfish14@gmail.com}{} \\
    } 
} 
\input{../templates/tabella_versioni.tex}
<<<<<<< HEAD
\newcommand{\sethdr}[1]{
		\pagestyle{fancy}
		\lhead{\includegraphics[width=1cm]{Swellfish_logo.png}}	
		\rhead{#1}
}

%\hypersetup{colorlinks=true,urlcolor=blue}

%\newcommand{\tableContent}{

	%{
		%\hypersetup{linkcolor=black}
		%\tableofcontents
	%}
=======
\newcommand{\sethdr}[1]{
		\pagestyle{fancy}
		\lhead{\includegraphics[width=1cm]{Swellfish_logo.png}}	
		\rhead{#1}
}


%\hypersetup{colorlinks=true,urlcolor=blue}

%\newcommand{\tableContent}{

	%{
		%\hypersetup{linkcolor=black}
		%\tableofcontents
	%}
>>>>>>> origin/piano_di_qualifica
%}

\begin{document}
\graphicspath{ {../templates/img/} }
\title{SWELLFish}

\begin{center}
    \includegraphics[width=0.3\textwidth]{../templates/img/Swellfish_logo.png}
    \hspace{3cm}
    \includegraphics[width=0.2\textwidth]{../templates/img/logoUnipd.png}\\
    \end{center}
    \begin{flushright}
        \
        \textbf{}\\
        28 Luglio 2023
    \end{flushright}  

Egregio Prof. Vardanega,\\\\
con la presente, il gruppo SWEllFish desidererebbe candidarsi in maniera ufficiale
per la fase di RTB.\\\\
Nel repository è presente una cartella RTB con all'interno due cartelle \textit{interni} ed \textit{esterni} contententi tutti i documenti necessari per la consegna.\\
Nel repository all'indirizzo: \\
\href{https://github.com/SWEllfish14/Documentazione}{\underline{https://github.com/SWEllfish14/Documentazione}}\\ 
potrà trovare i seguenti documenti:\\\\

Nella cartella \textit{RTB/documenti esterni}:
\begin{itemize}
\item piano\_di\_progetto\_v1.0.0;
\item piano\_di\_qualifica\_v1.0.0;
\item Lettera\_di\_presentazione;
\item Glossario\_v1.0.0;
\end{itemize}
Nella cartella \textit{RTB/documenti interni}:
\begin{itemize}
    \item norme\_di\_progetto\_v1.0.0;
\end{itemize}
 Nella cartella \textit{RTB/documenti interni/verbali} e nella cartella \textit{RTB/documenti esterni/verbali} sono inoltre presenti rispettivamente i verbali interni ed esterni prodotti dal gruppo. 

\begin{flushright}
Cordiali Saluti,\\
SWELLFish
\end{flushright}

\end{document}