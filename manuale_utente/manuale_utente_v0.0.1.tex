\documentclass[9pt]{article}
\usepackage{graphicx} % Required for inserting images
\usepackage{tabularx}
\usepackage{hyperref}
\usepackage{makecell}
\usepackage{makeidx}
\usepackage{eurosym}
\usepackage{fancyhdr}
\usepackage{titlesec}
\usepackage[export]{adjustbox}
\usepackage{float}
\graphicspath{ {./img/} }

\newcommand{\firstPage}{
    \begin{figure}
    \centering
    \includegraphics[scale=0.5]{Swellfish_logo.png}
    \end{figure}
    \author{Andrea Veronese, Claudio Giaretta, Elena Marchioro,\\
    Davide Porporati, Francesco Naletto, Jude Vensil Barceros \\ \\
    \href{swellfish14@gmail.com}{} \\
    } 
} 
\input{../templates/tabella_versioni.tex}
<<<<<<< HEAD
\newcommand{\sethdr}[1]{
		\pagestyle{fancy}
		\lhead{\includegraphics[width=1cm]{Swellfish_logo.png}}	
		\rhead{#1}
}

%\hypersetup{colorlinks=true,urlcolor=blue}

%\newcommand{\tableContent}{

	%{
		%\hypersetup{linkcolor=black}
		%\tableofcontents
	%}
=======
\newcommand{\sethdr}[1]{
		\pagestyle{fancy}
		\lhead{\includegraphics[width=1cm]{Swellfish_logo.png}}	
		\rhead{#1}
}


%\hypersetup{colorlinks=true,urlcolor=blue}

%\newcommand{\tableContent}{

	%{
		%\hypersetup{linkcolor=black}
		%\tableofcontents
	%}
>>>>>>> origin/piano_di_qualifica
%}

\begin{document}
\graphicspath{ {../templates/img/} {./img}}
\setcounter{tocdepth}{4}
\setcounter{secnumdepth}{4}
\title{Manuale Utente}

\firstPage
\maketitle

\begin{center}
	\begin{tabular}{r | l}
		\multicolumn{2}{c}{\textit{Informazioni}}        \\
		\hline

		\textit{Redattori}    &
		[Claudio Giaretta, Francesco Naletto]\makecell{} \\

		\textit{Revisori}     &
		[Jude Vensil Barceros]\makecell{}                \\
		\textit{Responsabili} &
		[Andrea Veronese]\makecell{}                     \\
		\textit{Uso}          &
		[Esterno]\makecell{}                             \\
	\end{tabular}
\end{center}

\begin{center}
	\textbf{Descrizione}\\
	File contenente Questo documento racchiude le istruzioni per l’utilizzo corretto di LumosMinima
\end{center}

\pagebreak

\printindex
\pagebreak

\tableofcontents
\pagebreak

\addversione{0.0.0}{23/08/2023}{Davide Porporati}{Claudio Giaretta}{Impostata struttura documento}
\addversione{0.0.1}{04/09/2023}{Francesco Naletto}{Claudio Giaretta}{Redatta prima versione documento}

\makeversioni

\section{Introduzione}
\subsection{Scopo del documento}
Il presente documento ha lo scopo di delineare le funzioni offerte dall'applicazione e di fornire istruzioni dettagliate per l'utente sull'uso di quest'ultima.
In tal modo, l'utente verrà informato sui requisiti minimi indispensabili per assicurare il corretto funzionamento di LumosMinima.
\subsection{Cos'è LumosMinima}
LumosMinima è un progetto che mira a ottimizzare l'illuminazione pubblica. L'obiettivo è regolare l'intensità luminosa degli impianti pubblici al fine di garantire sicurezza e risparmio energetico. Il sistema agisce automaticamente sui dispositivi di illuminazione, rileva la presenza umana, segnala guasti e offre un controllo globale dell'intensità luminosa. LumosMinima rappresenta una soluzione innovativa per affrontare le sfide energetiche e ambientali del futuro.

\section{Strumenti necessari}
Per garantire il corretto funzionamento dell'applicazione, è indispensabile avere una connessione internet attiva. L'applicazione è stata verificata e supporta i seguenti browser:

\begin{itemize}
	\item Chrome, versione 116.
	\item Safari, versione 16.5.
	\item Edge, versione 116.
	\item Firefox, versione 117.
	\item Opera, versione 102.
\end{itemize}

\section{LumosMinima}
\subsection{Home}
Il sito LumosMinima si presenta con una dashboard composta di 4 pannelli che danno informazioni generali
sul sistema

\begin{center}
	\includegraphics[scale=0.3]{LumosMinimaHome.png}
\end{center}

In particolare:
\begin{itemize}
	\item \textbf{la sezione 1}: indica informazioni riguardanti lo stato del sistema
	\item \textbf{la sezione 2}: riporta una lista delle ultime 9 segnalazioni di guasti inseriti a sistema dall'utente
	\item \textbf{la sezione 3}: riporta una lista delle aree illuminate inserite a sistema
	\item \textbf{la sezione 4}: riporta un link in cui è possibile scaricare il manuale utente\\
\end{itemize}

La navbar presente in alto (sezione 5) è composta da 3 link:

\begin{itemize}
	\item \textbf{Home}: Porta alla pagina corrente ovvero la Home
	\item \textbf{Gestione aree}: porta alla pagina "Gestione Aree" in cui è possibile gestire le varie aree inserite a sistema
	\item \textbf{Gestione guasti}: porta alla pagina in cui vengono visualizzati tutti i guasti segnalati a sistema
\end{itemize}

\subsection{Gestione aree}
In questa pagina è presente la lista di tutte le aree inserite a sistema.
Ogni area ha un tasto associato che porta alla pagina di dettaglio corrispondente all'area cliccata.

\begin{center}
	\includegraphics[scale=0.3]{Gestione_aree.png}
\end{center}

Nella sezione 2 è presente il tasto per aggiungere una nuova area di illuminazione, che porterà alla pagina "Aggiugni area", insieme ai due tasti per aumentare e diminuire la luminosità di tutte le aree.

\subsection{Aggiungi area}

In questa pagina è possibile immettere tutte le informazioni riguardanti l'area illuminata che si vuole inserire.

\begin{center}
	\includegraphics[scale=0.3]{Aggiungi_area.png}
\end{center}

Una volta terminato l'inserimento dei dati, cliccando sul pulsante "Conferma e Inserisci"
l'area verrà inserita nel sistema e si verrà reindirizzati nella pagina "Area" con il dettaglio dell'area appena inserita.

\subsection{Area}

In questa area è possibile gestire tutti gli aspetti legati all'area di illuminazione selezionata

\begin{center}
	\includegraphics[scale=0.3]{Area.png}
\end{center}

La pagina è suddivisa nei seguenti riquadri:
\begin{itemize}
	\item \textbf{la sezione 1}: Informazioni contententi i dettagli dell'area, è presente un pulsante con cui è possibile aggiungere un guasto, porta alla pagina "Aggiungi guasto".
	\item \textbf{la sezione 2}: Contiene i pulsanti che permettono di gestire l'illuminazione dell'area in maniera manuale oppure automatica.
	\item \textbf{la sezione 3}: Contiene i pulsanti per l'eliminazione dell'area o per la modifica delle sue informazioni.
	\item \textbf{la sezione 4}: Contiene i pulsanti per aggiungere un sensore all'area, che porta alla pagina "Aggiungi sensore", e per visualizzare la lista dei sensori dell'area, che porta alla pagina "Lista sensori".
	\item \textbf{la sezione 5}: Contiene i pulsanti per aggiungere un lampione all'area, che porta alla pagina "Aggiungi lampione", e per visualizzare la lista dei lampioni dell'area, che porta alla pagina "Lista lampioni".
\end{itemize}

\subsection{Aggiungi guasto}
In questa pagina è possibile immettere tutte le informazioni riguardanti il guasto che si vuole inserire.

\begin{center}
	\includegraphics[scale=0.3]{Aggiungi_guasto.png}
\end{center}

Una volta terminato l'inserimento dei dati, cliccando sul pulsante "Conferma e Inserisci"
il guasto verrà inserita nel sistema.


\subsection{Aggiungi sensore}
In questa pagina è possibile immettere tutte le informazioni riguardanti il sensore che si vuole inserire.

\begin{center}
	\includegraphics[scale=0.3]{Aggiungi_sensore.png}
\end{center}

Una volta terminato l'inserimento dei dati, cliccando sul pulsante "Conferma e Inserisci"
il sensore verrà inserita nel sistema.

\subsection{Lista sensori}
In questa pagina è presente la lista di tutti i sensori dell'area indicata.
Ogni sensore ha un tasto associato che porta alla pagina di modifica del sensore.

\begin{center}
	\includegraphics[scale=0.3]{Lista_sensori.png}
\end{center}

\subsection{Modifica sensore}
In questa pagina è possibile immettere tutte le informazioni riguardanti il sensore che si vuole modificare.

\begin{center}
	\includegraphics[scale=0.3]{Modifica_sensore.png}
\end{center}

Una volta terminato l'inserimento dei dati, cliccando sul pulsante "Conferma e Inserisci"
la modifica verrà inserita nel sistema.
E' anche presente il tasto per eliminare il sensore dal sistema.


\subsection{Aggiungi lampione}
In questa pagina è possibile immettere tutte le informazioni riguardanti il lampione che si vuole inserire.

\begin{center}
	\includegraphics[scale=0.3]{Aggiungi_lampione.png}
\end{center}

Una volta terminato l'inserimento dei dati, cliccando sul pulsante "Conferma e Inserisci"
il lampione verrà inserita nel sistema.

\subsection{Lista lampioni}
In questa pagina è presente la lista di tutti i lampioni dell'area indicata.
Ogni lampione ha un tasto associato per spegnerlo e uno che porta alla pagina di modifica del lampione.

\begin{center}
	\includegraphics[scale=0.3]{Lista_lampioni.png}
\end{center}


\subsection{Modifica lampione}
In questa pagina è possibile immettere tutte le informazioni riguardanti il lampione che si vuole modificare.

\begin{center}
	\includegraphics[scale=0.3]{Modifica_lampione.png}
\end{center}

Una volta terminato l'inserimento dei dati, cliccando sul pulsante "Conferma e Inserisci"
la modifica verrà inserita nel sistema.
E' anche presente il tasto per eliminare il lampione dal sistema.


\subsection{Gestione guasti}
In questa pagina è presente la lista di tutti i guasti inseriti a sistema.
Ogni guasto ha un tasto associato che porta alla pagina di dettaglio corrispondente al guasto cliccato.

\begin{center}
	\includegraphics[scale=0.3]{Gestione_guasti.png}
\end{center}

\subsection{Guasto}

In questa area è possibile gestire tutti gli aspetti legati al guasto selezionato.

\begin{center}
	\includegraphics[scale=0.3]{Guasto.png}
\end{center}

La pagina è suddivisa nei seguenti riquadri:
\begin{itemize}
	\item \textbf{la sezione 1}: Informazioni contententi i dettagli del guasto.
	\item \textbf{la sezione 2}: Contiene i pulsanti per la chiusura del guasto o per la modifica delle sue informazioni.
\end{itemize}

\subsection{Modifica guasto}
In questa pagina è possibile immettere tutte le informazioni riguardanti il guasto che si vuole modificare.

\begin{center}
	\includegraphics[scale=0.3]{Modifica_guasto.png}
\end{center}

Una volta terminato l'inserimento dei dati, cliccando sul pulsante "Conferma e Inserisci"
la modifica verrà inserita nel sistema.



\end{document}
