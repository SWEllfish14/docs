\documentclass[12pt]{article}
\usepackage{graphicx} % Required for inserting images
\usepackage{hyperref}
\usepackage{makecell}
\usepackage{makeidx}
\usepackage{eurosym}
\usepackage{fancyhdr}
\usepackage{titlesec}
\graphicspath{ {./img/} }

\newcommand{\firstPage}{
    \begin{figure}
    \centering
    \includegraphics[scale=0.5]{Swellfish_logo.png}
    \end{figure}
    \author{Andrea Veronese, Claudio Giaretta, Elena Marchioro,\\
    Davide Porporati, Francesco Naletto, Jude Vensil Barceros \\ \\
    \href{swellfish14@gmail.com}{} \\
    } 
} 
\input{../templates/tabella_versioni.tex}
<<<<<<< HEAD
\newcommand{\sethdr}[1]{
		\pagestyle{fancy}
		\lhead{\includegraphics[width=1cm]{Swellfish_logo.png}}	
		\rhead{#1}
}

%\hypersetup{colorlinks=true,urlcolor=blue}

%\newcommand{\tableContent}{

	%{
		%\hypersetup{linkcolor=black}
		%\tableofcontents
	%}
=======
\newcommand{\sethdr}[1]{
		\pagestyle{fancy}
		\lhead{\includegraphics[width=1cm]{Swellfish_logo.png}}	
		\rhead{#1}
}


%\hypersetup{colorlinks=true,urlcolor=blue}

%\newcommand{\tableContent}{

	%{
		%\hypersetup{linkcolor=black}
		%\tableofcontents
	%}
>>>>>>> origin/piano_di_qualifica
%}
\begin{document}
\graphicspath{ {../templates/img/} }
\setcounter{tocdepth}{4}
\setcounter{secnumdepth}{4}
\title{Analisi delle tecnologie}

\firstPage

\pagestyle{genericDocstyle}
\maketitle

\begin{center}
    \begin{tabular}{r | l}
		\multicolumn{2}{c}{\textit{Informazioni}}\\
		\hline
		
			\textit{Redattori} &
			[Andrea Veronese]\makecell{}\\

			\textit{Revisori} &
			[Claudio Giaretta]\makecell{}\\
			\textit{Responsabili} &
			[Davide Porporati]\makecell{}\\
		      \textit{Uso} & 
                [Esterno]\makecell{}\\
    \end{tabular}
\end{center}

\begin{center}
    \textbf{Descrizione}\\
	File contenente l'analisi delle tecnologie scelte per realizzare il progetto. 
\end{center}

\pagebreak

\tableofcontents

\printindex 

\addversione{0.0.0}{10/05/2023}{Andrea Veronese}{Claudio Giaretta}{Creata struttura di base del documento}
\addversione{0.0.1}{24/05/2023}{Claudio Giaretta}{Davide Porporati, Elena Marchioro}{Rework struttura con aggiunta analisi tecnologie}
\addversione{0.1.0.}{17/07/2023}{Andrea Veronese}{Davide Porporati, Claudio Giaretta}{Revisione del documento}

\makeversioni

\section{Introduzione}
Il progetto proposto da Imola Informatica verte sullo sviluppo di una WebApp che sia in grado di gestire l'illuminazione di lampioni smart, suddividendoli in zone di afferenza.
Il prodotto finale è quindi composto dalle seguenti parti:
\begin{itemize}
	\item WebApp che funge da dashboard, dove è possibile interagire graficamente con il sistema
	\item script di interfacciamento con i lampioni smart e con i sensori
	\item una schermata di login/logout
	\item un sistema di ticketing di terze parti, utilizzato per gestire i guasti di sistema
	\item un database per memorizzare i lampioni e i vari componenti utilizzati per gestire le funzionalità automatiche/manuali del sistema
\end{itemize}

\section{Frontend}
Per realizzare la parte frontend del prodotto il gruppo ha optato per l'utilizzo del framework React. \\
La scelta è ricaduta su quest framework perchè è open-source e non richiede il pagamento di canoni mensili, come richiesto da capitolato.
Oltre a react è stato valutato anche il framework Angular e le motivazioni che hanno portato a questa scelta sono riassumibili nella seguente tabella: \\
\begin{center}
\begin{tabular}{ |p{8cm}|p{8cm}| }
	\hline
	 \textbf{React} & \textbf{Angular}\\ 
	 \hline
	 Maggiore richiesta delle aziende & Maggiore pesantezza rispetto a React \\  
	 \hline
	 Utilizzo componenti nativi & Cross-Platform \\ 
	 \hline
	 Componenti riutilizzabili & Usa pattern MVC \\
	 \hline
	 Compatibilità con altri framework & Data-binding bidirezionale \\
	\hline
\end{tabular}
\end{center}
Nonostante la maggiore difficoltà di apprendimento di React, il gruppo ha comunque deciso di utilizzarlo data l'utilità futura in ambito lavorativo.

Si segnala inoltre che per gestire il CSS da utilizzare per lo stile grafico della WebApp, il team ha optato per l'utilizzo del framework "Bulma". I motivi che hanno portato alla seguente scelta sono riassumibili così:
\begin{itemize}
    \item  facilità d'uso del framework
    \item  notorioetà
    \item  alternativa valida ai framework indicati come standard "de-facto"
    \item  presenza di una documentazione ben fornita con esempi di utilizzo
\end{itemize} 
\section{Backend}
La scelta per lo sviluppo del backend è stata dettata puramente dalla preferenza verso un linguaggio staticamente o dinamicamente tipizzato.
I linguaggi che permettono la tipizzazione statica sono i seguenti:
\begin{itemize}
	\item Java
	\item \texttt{C\#}
	\item TypeScript
\end{itemize}
Per quanto riguarda la tipizzazione dinamica, le scelte possibili sono le seguenti:
\begin{itemize}
	\item Python
	\item JavaScript
\end{itemize}

I vari framework disponibili per la tipizzazione statica sono riportati su questa tabella: \\
\begin{center}
\begin{tabular}{ |p{4cm}|p{4cm}| p{4cm} | }
	\hline
	 \textbf{Java} & \texttt{C\#} & \textbf{TypeScript}\\ 
	 \hline
	 JavaSpring & ASP.NET & Nest \\  
	 \hline
 	& & Feathers\\ 
	 \hline
	 & & loopback\\
	 \hline
\end{tabular}
\end{center}

I vari framework disponibili per la tipizzazione dinamica sono riportati su questa tabella: \\

\begin{tabular}{ |p{5cm}|p{5cm}| }
	\hline
	 \textbf{Python} & \textbf{JavaScript}\\ 
	 \hline
	 Django & Next \\  
	 \hline
 	Flask & Node\\ 
	 \hline
\end{tabular}
\end{center}

\subsection{Database}
Per le implementare la persistenza dei dati le scelte possibili sono sostanzialmente due: un database standard SQL o un'alternativa NoSQL.
Entrambe sono ben accettate dall'azienda proponente e il gruppo ha optato per la versione SQL.
In particolare si è deciso di utilizzare MariaDB, in quanto non ci sono particolari necessità da soddisfare e la documentazione è ben fornita e completa.

\begin{center}
	\begin{tabular}{ |p{8cm}|p{8cm}| }
		\hline
		 \textbf{SQL} & \textbf{NoSQL}\\ 
		 \hline
		 Schema tabellare rigido & Assenza modello dati \\  
		 \hline
		 General Purpose &Schema dinamico per dati non strutturati\\ 
		 \hline
		 Scalabilità Verticale & Utilizzo modelli di dati non tabulari \\
		 \hline
\end{tabular}
\end{center}
Pur essendo il modello NoSQL più flessibile, il team ha comunque deciso di utilizzare un database SQL visto che conosce già il funzionamento e le modalità di sviluppo del database.
Così facendo è possibile risparmiare del tempo nell'apprendimento del nuovo modello fornito dai database NoSQL e impiegarlo per lo studio dei framework impiegati nello sviluppo delle altre componenti del progetto.

\end{document}
