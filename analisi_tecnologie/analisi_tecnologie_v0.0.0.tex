\documentclass[12pt]{article}
\usepackage{graphicx} % Required for inserting images
\usepackage{hyperref}
\usepackage{makecell}
\usepackage{makeidx}
\usepackage{eurosym}
\usepackage{fancyhdr}
\usepackage{titlesec}
\graphicspath{ {./img/} }

\newcommand{\firstPage}{
    \begin{figure}
    \centering
    \includegraphics[scale=0.5]{Swellfish_logo.png}
    \end{figure}
    \author{Andrea Veronese, Claudio Giaretta, Elena Marchioro,\\
    Davide Porporati, Francesco Naletto, Jude Vensil Barceros \\ \\
    \href{swellfish14@gmail.com}{} \\
    } 
} 
\input{../templates/tabella_versioni.tex}
<<<<<<< HEAD
\newcommand{\sethdr}[1]{
		\pagestyle{fancy}
		\lhead{\includegraphics[width=1cm]{Swellfish_logo.png}}	
		\rhead{#1}
}

%\hypersetup{colorlinks=true,urlcolor=blue}

%\newcommand{\tableContent}{

	%{
		%\hypersetup{linkcolor=black}
		%\tableofcontents
	%}
=======
\newcommand{\sethdr}[1]{
		\pagestyle{fancy}
		\lhead{\includegraphics[width=1cm]{Swellfish_logo.png}}	
		\rhead{#1}
}


%\hypersetup{colorlinks=true,urlcolor=blue}

%\newcommand{\tableContent}{

	%{
		%\hypersetup{linkcolor=black}
		%\tableofcontents
	%}
>>>>>>> origin/piano_di_qualifica
%}
\begin{document}
\graphicspath{ {../templates/img/} }
\setcounter{tocdepth}{4}
\setcounter{secnumdepth}{4}
\title{Analisi delle tecnologie}

\firstPage

\pagestyle{genericDocstyle}
\maketitle

\begin{center}
    \begin{tabular}{r | l}
		\multicolumn{2}{c}{\textit{Informazioni}}\\
		\hline
		
			\textit{Redattori} &
			[Andrea Veronese]\makecell{}\\

			\textit{Revisori} &
			[Claudio Giaretta]\makecell{}\\
			\textit{Responsabili} &
			[Davide Porporati]\makecell{}\\
		      \textit{Uso} & 
                [Esterno]\makecell{}\\
    \end{tabular}
\end{center}

\begin{center}
    \textbf{Descrizione}\\
	File contenente l'analisi delle tecnologie scelte per realizzare il progetto. 
\end{center}

\pagebreak

\tableofcontents
\pagebreak

\printindex 

\addversione{0.0.0}{10/05/2023}{Andrea Veronese}{Claudio Giaretta}{Creata struttura di base del documento}

\makeversioni

\section{Introduzione}
Il progetto proposto da Imola Informatica verte sullo sviluppo di una WebApp che sia in grado di gestire l'illuminazione di lampioni smart, suddividendoli in zone di afferenza.
Il prodotto finale è quindi composto dalle seguenti parti:
\begin{itemize}
	\item WebApp che fornisca da dashboard, dove è possibile gestire ogni aspetto dell'applicazione
	\item script di interfacciamento con i lampioni smart e con i sensori
	\item una schermata di login/logout
	\item un sistema molto basilare di ticketing per gestire i guasti di sistema
	\item un database per memorizzare i lampioni e i vari componenti utilizzati per comandarne il funzionamento
	\item cifratura delle comunicazioni, in modo che i comandi vengano eseguiti senza interferenze esterne
\end{itemize}

\section{Frontend}
Per realizzare la parte frontend del prodotto il gruppo ha optato per l'utilizzo del framework Angular. La scelta è ricaduta su quest framework perchè è open-source e non richiede il pagamento di canoi mensili, come richiesto da capitolato d'appalto.

\section{Backend}
Il backend verrà gestito tramite JavaScript e PHP
\section{Interfacciamento con i lampioni}
Per l'interfacciamento con i lampioni la scelta è ricaduta sull'utilizzo delle api rest.
Tali api verranno testate con tool disponibili gratuitamente come:
\begin{itemize}
    \item Talend API tester 
\end{itemize}
L'interfacciamento con le API verrà realizzato in Python
\end{document}
