\documentclass[12pt]{article}
\usepackage{graphicx} % Required for inserting images
\usepackage{hyperref}
\usepackage{makecell}
\usepackage{makeidx}
\usepackage{eurosym}
\graphicspath{ {./img/} }

\newcommand{\firstPage}{
    \begin{figure}
    \centering
    \includegraphics[scale=0.5]{Swellfish_logo.png}
    \end{figure}
    \author{Andrea Veronese, Claudio Giaretta, Elena Marchioro,\\
    Davide Porporati, Francesco Naletto, Jude Vensil Barceros \\ \\
    \href{swellfish14@gmail.com}{} \\
    } 
}
\input{../templates/tabella_versioni.tex}
\begin{document}

\graphicspath{ {../templates/img/} }

\title{Norme di progetto}

\firstPage

\maketitle

\begin{center}
    \begin{tabular}{r | l}
		\multicolumn{2}{c}{\textit{Informazioni}}\\
		\hline
		
			\textit{Redattori} &
			[Andrea Veronese]\makecell{}\\

			\textit{Revisori} &
			[Davide Porporati]\makecell{}\\
			\textit{Responsabili} &
			[Elena Marchioro]\makecell{}\\
		      \textit{Uso} & 
                [Interno]\makecell{}\\
    \end{tabular}
\end{center}

\begin{center}
    \textbf{Descrizione}\\
    File contenente tutte le best practices applicate al progetto 
\end{center}

\pagebreak

\tableofcontents
\pagebreak

\printindex 

\addversione{0.0.0}{22/03/2023}{Andrea Veronese}{Davide Porporati}{Versione preliminare con convenzioni di base}
\addversione{0.0.1}{19/04/2023}{Andrea Veronese}{Davide Porporati}{Aggiunta convenzioni adottate per lo sviluppo}
\addversione{0.0.2}{23/04/2023}{Davide Porporati, Elena Marchioro, Francesco Naletto}{Jude Vensil Barceros}{Aggiunte sezioni verifica e scrum}
\makeversioni
\section{Introduzione}
Lo scopo del documento è quello di fornire un insieme di regole, strumenti e stadard di qualità necessari per creare un Way of Working condiviso e efficacemente impiegabile da tutti i membri del gruppo.
Questo documento conterrà tutte le convenzioni da utilizzare per lo sviluppo del progetto e verrà aggiornato in maniera incrementale durante tutte le fasi dello sviluppo, pertanto è da considerarsi come work-in-progress.


\section{Processi Primari}
\subsection{Repository GitHub}
Data la necessità di effettuare un lavoro sostanzioso e condiviso, il team ha deciso di utilizzare una repository pubblica su GitHub per rendere agevole la collaborazione e la visione delle modifiche apportate alle varie parti del progetto.
E' stata quindi creata un'organizzazione chiamata SWEllFish e al suo interno sono state create delle specifiche repository:

\begin{itemize}
    \item SWELLFish: questa repo conterrà tutto il codice necessario per il progetto
    \item Documentazione: questa repo contiene tutta la documentazione di interesse per il committente e il proponente.
    \item dev: repository che contiene i file sorgenti neccessari per la documentazione
\end{itemize}

\subsection{Contatti con il committente/proponente}
Per semplificare la gestione dei contatti e delle eventuali richieste con l'azienda committente e con i Professori Vardenega e Cardin, è stata creata una mail, swellfish14@gmail.com, utilizzabile da tutti i componenti del gruppo.
L'uso di questa casella è stato regolamentato con un insieme di norme condivise dagli sviluppatori, pertanto un messaggio nasce dalla necessità di uno o più componenti del gruppo di avere delucidazioni da parte del committente o per sottoporre una modifica in attesa di accettazione.
Il canale telegram verrà usato per chiedere informazioni in modo informale con il proponente oppure per richiedere un confronto veloce.
I colloqui vengono richiesti quando le domande da porre necessitano di una discussione diretta con il proponente.
Questi ultimi vengono richiesti via mail oppure via telegram, tramite l'apposito gruppo creato con il responsabile incaricato di Imola Informatica.





\section{Scrum}
Il team utilizzerà questo framework per gestire le varie fasi di sviluppo del progetto.
\subsection{Organizzazione degli Sprint}
Ogni sprint avrà le seguenti caratteristiche:
\begin{itemize}
    \item Durata: una settimana. Inizia il venerdì e termina il venerdì successivo con la riunione concordata
    \item Riunione: viene effettuata il venerdì al termine dello sprint e vengono discussi i seguenti punti:
    \begin{itemize}
        \item Resoconto delle attività completate
        \item Problemi riscontrati e soluzioni adottate
        \item Analisi e resoconto delle ore e dei costi sostenuti
        \item Pianificazione data inizio prossimo sprint e obiettivi da fissare
    \end{itemize}
\end{itemize}
Gli obiettivi pianificati per lo sprint sucessivo verranno tradotti in "item" da aggiungere alla/e board presenti su Github e ad ogni item verranno assegnati dei membri del gruppo, coerentemente con la suddivisione dei ruoli per lo sprint in questione.
    

\section{Rotazione dei ruoli}
Vista l'adozione del framework scrum, il team ha deciso di far ruotare i ruoli dei componenti in corrispondenza della fine dello sprint, seguendo l'ordine alfabetico del nome dei componenti.
Pertanto ogni membro del team avrà impegni e responsabilità diverse ogni settimana. Questo approccio permette di ottenere un maggiore coinvolgimento.


\section{Processi di supporto}
\subsection{Organizzazione delle task}
Per organizzare meglio il lavoro e avere una panoramica chiara dello svolgimento delle attività e dello stato di avanzamento, è stato concordato l'uso della Project Board fornita da GitHub.
Le task riguardanti la documentazione e lo svolgimento del progetto vengo assegnate ai componenti dall'amministratore ed egli si occupa inoltre di chiuderle o modificarle in base alle esigenze.
Per agevolare l'attività di documentazione è stata creata una board denominata "To-Do Documenti", dove in concomitanza dello sprint settimanale vengono aggiunte nuove issue riguardanti la creazione o la modifica dei documenti da affrontare nel prossimo sprint.
Per gestire le attività di sviluppo verrà utilizzata un ulteriore board chiamata "SwellFish", al cui interno verranno inserite le task da completare in un dato sprint.
Queste board sono raggiungibili tramite l'apposita sezione "Projects" all'interno dell'organizzazione SwellFish.

\subsection{Documentazione}
Ogni documento che verrà redatto attraverserà diversi processi, ognuno dei quali vedrà impegnati uno o più componenti del gruppo.
Di seguito diamo una visione d'insieme delle fasi che ogni documento attraverserà:
\begin{itemize}
    \item Pianificazione: il file viene ideato e discusso con i membri del team sulla base delle necessità che deve soddisfare
    \item Stesura: il testo e il contenuto del documento viene realizzato dal redattore
    \item Revisione: il revisore incaricato si occupa di rileggerlo e correggere eventuali errori e/o imprecisioni, controllando che siano state rispettate le convenzioni stabilite in questo documento.
    \item Approvazione: il responsabile designato controlla che il documento sia corretto e coerente.
\end{itemize}
\section{Organizzazione dei File}
Tutta la documentazione inerente al progetto sarà incusa nella repository "Documentazione" all'interno dell'organizzazione chiamata "SwellFish".
I documenti di interesse per il proponente/committente si trovano in Documentazione/esterni, mentre quelli necessari al team di sviluppo si trovano in Documentazione/interni. 
I verbali infine sono contenuti in Documentazione/esterni/verbali.
\subsection{Strumenti impiegati}
Per la redazione della documentazione il team ha concordato l'uso del linguaggio Latex.
Per facilitare la produzione di documenti e ridurre i tempi necessari per la redazione, sono stati ideati dei template generici da utilizzare.
Tali templates sono reperibili in "$docs/tree/dev/templates$"
\subsection{Struttura di un documento}
Ogni documento che non sia un verbale ha una struttura ben specifica:
\begin{itemize}
    \item Nome del gruppo
    \item Logo 
    \item Tabella con le modifiche fatte e relativo numero di versionamento (se previsto)
    \item Tabella con ruoli dei componenti del gruppo e finalità d'uso del documento
    \item Indice
    \item Breve descrizione dello scopo del documento
\end{itemize}
\subsection{Struttura di un verbale}
La struttura di un verbale utilizza la struttura di base del template impiegato per la stesura di un documento, con le seguenti aggiunte
\begin{itemize}
    \item Data
    \item Ora
    \item Durata
    \item Partecipanti
    \item Ordine del giorno
    \item Riassunto contenuti
\end{itemize}
Per la nomenclatura dei verbali invece segue la seguente struttura:
\begin{itemize}
    \item verbale
    \item data
    \item nome azienda/docente/attività svolta
\end{itemize}

\section{Versionamento}
Il VCS scelto dal gruppo per facilitare la gestione delle modifiche è Git, mediante l'utilizzo di GitHub.
A questo scopo è stata creata un'organizzazione denominata "SwellFish", al cui interno sono state create le opportune repository in base alle necessità del progetto.
Le varie repository presenti nell'organizzazione sono raggiungibili al seguente link : "$https://github.com/orgs/SWEllfish14/repositories$".
\subsection{Regole per il versionamento}
Come prassi per la nomenclatura dei file il gruppo ha concordato la seguente struttura condivisa:
\begin{itemize}
    \item nome file: tutto in minuscolo, con le parole separate dall'underscore
    \item numero versione
\end{itemize}

Per la verifica e la validazione dei file il gruppo ha deciso di utilizzare la seguente prassi:
\begin{itemize}
    \item Creazione di un nuovo branch per un apposito documento
    \item Stesura del documento
    \item Caricamento sul branch appena creato
    \item Verifica della struttura, contenuti e rispetto delle convenzioni stabilite
    \item Pull request sul branch dev
    \item Rilascio sul branch master
\end{itemize}

\subsection{Regole per il versionamento}
E' stato concordato l'uso del Versionamento Semantico, affindandosi alle caratteristiche principali.
La base del versionamento ha struttura "X.Y.Z" e le seguenti regole verranno usate per indicare una modifica minore o sostanziale del documento.
\begin{itemize}
    \item Major Zero: E' la versione 0.X.Y, denota una versione preliminare del documento
    \item Patch Z: un aumento del valore della "Z" indica che sono state introdotte delle modifiche al testo da parte del redattore
    \item Minor Y: l'aumento della 'Y' indica una modifica testuale che è stata approvata dal verificatore
    \item Major X: l'incremento della 'X' si ha se è stata pubblicata dal responsabile una modifica non compatibile con le versioni precedenti del documento
\end{itemize}

\subsection{Comandi utili di Git}
I comandi più frequentemente utilizzati sono riassunti in seguito:
\begin{itemize}
    \item Sincronizzazione con repository remota: git pull;
    \item Creazione di un nuovo branch: git branch nome branch;
    \item Passaggio a un branch specificata: git checkout nome branch;
    \item Aggiunta delle modifiche alla stage area: git add *;
    \item Creazione del commit con le modifiche: git commit;
    \item Push sul remote di un nuovo branch: git push --set-upstream origin nome branch;
    \item Push su un branch già esistente: git push
\end{itemize}
\subsection{Regole per la verifica dei documenti}
Il verificatore incaricato utilizzerà la seguente procedura per valutare l'idoneità di un documento:
\begin{itemize}
    \item Controllo della pull request da parte degli altri membri del gruppo
    \item Viene verificato che il documento prodotto si attenga alle linee guida presenti in questo documento
    \item Viene verificata la qualità del contenuto 
    \item Se i precedenti punti sono soddisfatti viene effettuato il merge nel main branch della repo Documentazione
\end{itemize}

\end{document}
