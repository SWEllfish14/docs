\documentclass[12pt]{article}
\usepackage{graphicx} % Required for inserting images
\usepackage{hyperref}
\usepackage{makecell}
\usepackage{makeidx}
\usepackage{eurosym}
\usepackage{fancyhdr}
\usepackage{titlesec}
\graphicspath{ {./img/} }

\newcommand{\firstPage}{
    \begin{figure}
    \centering
    \includegraphics[scale=0.5]{Swellfish_logo.png}
    \end{figure}
    \author{Andrea Veronese, Claudio Giaretta, Elena Marchioro,\\
    Davide Porporati, Francesco Naletto, Jude Vensil Barceros \\ \\
    \href{swellfish14@gmail.com}{} \\
    } 
} 
\input{../templates/tabella_versioni.tex}
<<<<<<< HEAD
\newcommand{\sethdr}[1]{
		\pagestyle{fancy}
		\lhead{\includegraphics[width=1cm]{Swellfish_logo.png}}	
		\rhead{#1}
}

%\hypersetup{colorlinks=true,urlcolor=blue}

%\newcommand{\tableContent}{

	%{
		%\hypersetup{linkcolor=black}
		%\tableofcontents
	%}
=======
\newcommand{\sethdr}[1]{
		\pagestyle{fancy}
		\lhead{\includegraphics[width=1cm]{Swellfish_logo.png}}	
		\rhead{#1}
}


%\hypersetup{colorlinks=true,urlcolor=blue}

%\newcommand{\tableContent}{

	%{
		%\hypersetup{linkcolor=black}
		%\tableofcontents
	%}
>>>>>>> origin/piano_di_qualifica
%}

\begin{document}


\graphicspath{ {../templates/img/} }
\setcounter{tocdepth}{4}
\setcounter{secnumdepth}{4}
\title{Norme di progetto}

\firstPage

\sethdr{Norme di progetto}
\maketitle

\begin{center}
    \begin{tabular}{r | l}
        \multicolumn{2}{c}{\textit{Informazioni}}      \\
        \hline

        \textit{Redattori}    &
        [Andrea Veronese][Claudio Giaretta]\makecell{} \\

        \textit{Revisori}     &
        [Davide Porporati]\makecell{}                  \\
        \textit{Responsabili} &
        [Elena Marchioro]\makecell{}                   \\
        \textit{Uso}          &
        [Interno]\makecell{}                           \\
    \end{tabular}
\end{center}

\begin{center}
    \textbf{Descrizione}\\
    File contenente tutte le best practices applicate al progetto
\end{center}

\pagebreak

\tableofcontents

\pagebreak

\printindex

\addversione{0.0.0}{22/03/2023}{Andrea Veronese}{Davide Porporati}{Versione preliminare con convenzioni di base}
\addversione{0.0.1}{19/04/2023}{Andrea Veronese}{Davide Porporati}{Aggiunta convenzioni adottate per lo sviluppo}
\addversione{0.0.2}{23/04/2023}{Davide Porporati, Elena Marchioro, Francesco Naletto}{Jude Vensil Braceros}{Aggiunte sezioni verifica e scrum}
\addversione{0.1.0}{04/05/2023}{Claudio Giaretta}{Jude Vensil Braceros}{Verifica generale del documento}
\addversione{0.1.1}{06/06/2023}{Jude Vensil Braceros}{Francesco Naletto}{Correzione di alcuni errori grammaticali}
\makeversioni
\section{Introduzione}
Lo scopo del documento è quello di fornire un insieme di regole, strumenti e standard di qualità necessari per creare un Way of Working condiviso ed efficacemente impiegabile da tutti i membri del gruppo.
Questo documento conterrà tutte le convenzioni da utilizzare per lo sviluppo del progetto e verrà aggiornato in maniera incrementale durante tutte le fasi dello sviluppo, pertanto è da considerarsi come work-in-progress.


\section{Processi Primari}
\subsection{Fornitura}
\subsubsection{Documenti}
I documenti da redarre per la fornitura sono:
\begin{itemize}
    \item Piano di progetto
    \item Piano di qualifica
\end{itemize}
\paragraph{Piano di progetto}
Il piano di progetto è il documento, redatto dagli amministratori e dal responsabile del progetto, che definisce l'approccio e le attività che saranno utilizzate per gestire e completare un progetto con successo.
Il documento è diviso nelle seguenti parti:
\begin{itemize}
    \item Analisi dei rischi e le strategie di mitigazione
    \item Modello di sviluppo
    \item Pianificazione
    \item Preventivo
    \item Consuntivo di periodo
\end{itemize}
\paragraph{Piano di qualifica}
Il piano di qualifica è il documento, redatto dal verificatore, che definisce le attività e le risorse necessarie per garantire che un prodotto, un servizio o un processo soddisfi i requisiti di qualità specificati. Il piano di qualifica descrive in dettaglio le attività che devono essere svolte, le responsabilità dei membri del team e le risorse necessarie per raggiungere gli obiettivi di qualità.
Il documento è formato dalle seguenti parti:
\begin{itemize}
    \item  Qualità dei processi
    \item  Qualità del prodotto
    \item  Specifica dei test
    \item 	Resoconto attività di verifica
\end{itemize}
\subsubsection{Repository GitHub}
Data la necessità di effettuare un lavoro sostanzioso e condiviso, il team ha deciso di utilizzare una repository pubblica su GitHub per rendere agevole la collaborazione e la visione delle modifiche apportate alle varie parti del progetto.
E' stata quindi creata un'organizzazione chiamata SWEllFish e al suo interno sono state create delle specifiche repository:

\begin{itemize}
    \item SWELLFish: questa repository conterrà tutto il codice necessario per il progetto
    \item Documentazione: questa repository contiene tutta la documentazione di interesse per il committente e il proponente.
    \item dev: repository che contiene i file sorgenti neccessari per la documentazione
\end{itemize}

\subsubsection{Contatti con il committente/proponente}
Per semplificare la gestione dei contatti e delle eventuali richieste con l'azienda committente e con i Professori Vardenega e Cardin, è stata creata una mail, swellfish14@gmail.com, utilizzabile da tutti i componenti del gruppo.
L'uso di questa casella è stato regolamentato con un insieme di norme condivise dagli sviluppatori, pertanto un messaggio nasce dalla necessità di uno o più componenti del gruppo di avere delucidazioni da parte del committente o per sottoporre una modifica in attesa di accettazione.
Il canale telegram verrà usato per chiedere informazioni in modo informale con il proponente oppure per richiedere un confronto veloce.
I colloqui vengono richiesti quando le domande da porre necessitano di una discussione diretta con il proponente.
Questi ultimi vengono richiesti via mail oppure via telegram, tramite l'apposito gruppo creato con il responsabile incaricato di Imola Informatica.

\subsection{Sviluppo}
\subsubsection{Attività}
La parte di sviluppo è suddivisa nelle seguenti attività:

\begin{itemize}
    \item Analisi dei requisiti
    \item Progettazione
    \item Codifica
\end{itemize}

\paragraph{Analisi dei requisiti}

\paragraph{Progettazione}
Durante fase di progettazione vengono definite le funzionalità, la struttura e i comportamenti del prodotto, appoggiandosi dell'analisi dei requisiti fatta preventivamente.

Inizialmente tutto ciò si tradurrà in un proof of Concept, ossia una prima demo per dimostrare la fattibilità del prodotto, e in seguito approfondita e descritta nel documento tecnico allegato alla Product Baseline.

La progettazione è divisa in due fasi:

\subparagraph{Requirements \& Technology Baseline}
I documenti neccessari in questa fase sono:
\begin{itemize}
    \item Piano di progetto
    \item Piano di qualifica
    \item Norme di progetto
    \item Verbali interni ed esterni
\end{itemize}

\subparagraph{Product Baseline}
I documenti necessari in questa fase sono:
\begin{itemize}
    \item Manuale utente
    \item Allegato tecnico
    \item Verbali di periodo
\end{itemize}

\paragraph{Codifica}
La codifica ha lo scopo di concretizzare il prodotto tramite la programmazione

\subparagraph{Stile della codifica}
\begin{itemize}
    \item \textbf{Variabili}: Le variabili devono seguire la notazione camelCase, iniziando quindi con la lettera minuscola e se composte da più parole, ogni parola successiva alla prima dovrà iniziare con la maiuscola
    \item \textbf{Metodi}: I metodi, devono seguire la stessa nomenclatura (camelCase) delle variabili
    \item \textbf{Indentazione}: Tutto il codice deve essere indentato seguendo le regole di indentazione del linguaggio di programmazione scelto
    \item \textbf{Commenti}: Nel caso il codice venga ritenuto di difficile comprensione il programmatore può inserire un commento per esplicitarlo. Tutti i commenti vanno inseriti la riga sopra del codice da commentare, in modo da rendere subito visibile il commento e renderlo di facile lettura
    \item \textbf{Blocchi condizionali}: tutti i blocchi condizionali devono essere racchiusi all'interno delle parentesi grafe, anche quando non sono necessarie, questo per rendere esplicito l'inizio e la fine di un blocco
    \item \textbf{Classi}: tutte le classi devono avere la lettera iniziale maiuscola, se composte da più parole ogni parola successiva dovrà anch'essa avere la prima lettera maiuscola
    \item \textbf{File}: Dovrà avere la lettera maiuscola ed un nome significativo rispetto al suo contenuto
\end{itemize}

Il team ha scelto l'inglese come lingua ufficiale per ogni componente del codice (commenti, variabili, classi ecc..)

\subparagraph{Strumenti per la codifica}
\begin{itemize}
    \item \textbf{Angular}:Un framework per la realizzazione applicazioni Web\\
          \begin{center}
              \url{https://angular.io/docs}
          \end{center}
    \item \textbf{Spring}: Un popolare framework di Java per la gestione della parte backend
          \begin{center}
              \url{https://spring.io/}
          \end{center}
\end{itemize}

\subsubsection{Workflow GitHub}
Tutti i documenti presenti nella repository docs, attraversano il seguente workflow:

\begin{itemize}
    \item Creazione del branch remoto per il rispettivo documento
    \item Lavoro in locale con successivo push sul branch remoto corrispondente
    \item Pull request tramite piattaforma web Gihtub per merge su branch dev
    \item Viene verificato che il documento prodotto si attenga alle linee guida presenti in questo documento
    \item Viene verificata la qualità del contenuto
    \item Se i precedenti punti sono soddisfatti, la pull request viene approvata dal responsabile e viene effettuato il merge nel branch dev della repo Documentazione
\end{itemize}

\section{Processi di supporto}

\subsection{Documentazione}
\subsubsection{Ciclo di vita del documento}
Ogni documento che verrà redatto attraverserà diversi processi, ognuno dei quali vedrà impegnati uno o più componenti del gruppo.
Di seguito diamo una visione d'insieme delle fasi che ogni documento attraverserà:
\begin{itemize}
    \item Pianificazione: il file viene ideato e discusso con i membri del team sulla base delle necessità che deve soddisfare
    \item Stesura: il testo e il contenuto del documento viene realizzato dal redattore
    \item Revisione: il revisore incaricato si occupa di rileggerlo e correggere eventuali errori e/o imprecisioni, controllando che siano state rispettate le convenzioni stabilite in questo documento.
    \item Approvazione: il responsabile designato controlla che il documento sia corretto e coerente.
\end{itemize}

\subsubsection{Strumenti utilizzati}
Di seguito gli strumenti utilizzati dal gruppo per la stesura della documentazione:
\begin{itemize}
    \item \textbf{Latex}: linguaggio di Markup per la creazioni di testi. E' stato scelto in quanto permette la stesura dei documenti con una struttura formale e standardizzata, oltre alla moltitudine di strumenti che mette a disposizione il linguaggio che facilitano la stesura della documentazione
          \begin{center}
              \url{https://www.latex-project.org/}
          \end{center}
    \item \textbf{Visual Studio Code e Latex workshop}: Editor popolare per la scrittura del codice in vari linguaggi di programmazione, utilizzato insieme al framework Latex WorkShop che permette di operare anche su codice latex. E' stato preferito a Overleaf in quanto permette di lavorare in locale e facilita l'interazione con sistemi di versionamento remoto come GitHub
          \begin{center}
              \url{https://code.visualstudio.com/docs}
          \end{center}
\end{itemize}


\subsubsection{Template}
Il team ha deciso di utilizzare dei template scritti in Latex per facilitare il lavoro ed evitare codice ridondante tra i vari file della documentazione.
Di seguito sono indicati i template utilizzati dal gruppo:
\begin{itemize}
    \item \textbf{first\_page}: template presente in tutti i documenti per la creazione della prima pagina
    \item \textbf{styles}: template presente in tutti i documenti che contiene dettagli di stile comuni a tutti i file della documentazione
    \item \textbf{tabella\_versioni}: template per facilitare la creazione della tabella delle versioni nei documenti in cui è previsto versionamento.

\end{itemize}
Tutti i template si possono trovare all'interno della repository nella cartella \url{https://github.com/SWEllfish14/docs/tree/dev/templates}

\subsubsection{Struttura del documento}
Ogni documento che non sia un verbale ha una struttura ben specifica:
\begin{itemize}
    \item Nome del gruppo
    \item Logo
    \item Tabella con le modifiche fatte e relativo numero di versionamento (se previsto)
    \item Tabella con ruoli dei componenti del gruppo e finalità d'uso del documento
    \item Indice
    \item Breve descrizione dello scopo del documento
\end{itemize}

\paragraph{Struttura delle pagine}
\begin{itemize}
    \item Intestazione:
          \begin{itemize}
              \item Lato sinistro: logo del gruppo
              \item Lato destro: nome del documento
          \end{itemize}
    \item Corpo del documento:  strutturato in sezioni e sottosezioni, ognuna con il proprio titolo e sottotitolo.Il corpo del documento include anche qualsiasi supporto visivo, come tabelle, figure, grafici, 	diagrammi o illustrazioni, che aiutano a presentare le informazioni in modo più accessibile o coinvolgente.
    \item Piè di pagina: al centro è presente il numero di pagina.
\end{itemize}

\paragraph{Registro delle versioni}
Ogni file, fatta esclusione per i verbali, contiene una tabella che riporta le modifiche apportate al file nel corso del tempo.
La tabella è così strutturata:
\begin{itemize}
    \item \textbf{Versione}: versione attuale del documento (la nomenclatura segue le \hyperref[ver]{Regole per il versionamento})
    \item \textbf{Data}: data in cui viene rilasciata la corrispondente versione del documento
    \item \textbf{Redattore}: Indica chi ha apportato i cambiamenti al documento
    \item \textbf{Verificatore}: Indica chi si è occupato della verifica del documento
    \item \textbf{Descrizione}: Breve descrizione della modifica apportata nella corrispondente versione
\end{itemize}
Per facilitare l'inserimento della tabella all'interno dei vari file è stato creato un template apposito come spiegato nella sezione \hyperref[ver]{Template}

\subsubsection{Struttura di un verbale}
La struttura di un verbale utilizza la struttura di base del template impiegato per la stesura di un documento, con le seguenti aggiunte
\begin{itemize}
    \item Data
    \item Ora
    \item Durata
    \item Partecipanti
    \item Ordine del giorno
    \item Riassunto contenuti
\end{itemize}
\paragraph{Nomenclatura verbali}
Per la nomenclatura dei verbali invece segue la seguente struttura:
\begin{itemize}
    \item verbale
    \item nome azienda/docente/attività svolta
    \item data
\end{itemize}

\subsection{Gestione della configurazione}
\subsubsection{Versionamento}
\paragraph{Sistemi software utilizzati}
Il VCS scelto dal gruppo per facilitare la gestione delle modifiche è Git, mediante l'utilizzo di GitHub.
A questo scopo è stata creata un'organizzazione denominata "SwellFish", al cui interno sono state create le opportune repository in base alle necessità del progetto.
Le varie repository presenti nell'organizzazione sono raggiungibili al seguente link : "$https://github.com/orgs/SWEllfish14/repositories$".

\paragraph{Codice di versione}
Come prassi per la nomenclatura dei file il gruppo ha concordato la seguente struttura condivisa:
\begin{itemize}
    \item nome file: tutto in minuscolo, con le parole separate dall'underscore
    \item numero versione
\end{itemize}

Per la verifica e la validazione dei file il gruppo ha deciso di utilizzare la seguente prassi:
\begin{itemize}
    \item Creazione di un nuovo branch per un apposito documento
    \item Stesura del documento
    \item Caricamento sul branch appena creato
    \item Verifica della struttura, contenuti e rispetto delle convenzioni stabilite
    \item Pull request sul branch dev
    \item Rilascio sul branch master
\end{itemize}

\paragraph{Regole per il versionamento}
\label{ver}
E' stato concordato l'uso del Versionamento Semantico, affidandosi alle caratteristiche principali.
La base del versionamento ha struttura "X.Y.Z" e le seguenti regole verranno usate per indicare una modifica minore o sostanziale del documento.
\begin{itemize}
    \item Major Zero: E' la versione 0.X.Y, denota una versione preliminare del documento
    \item Patch Z: un aumento del valore della "Z" indica che sono state introdotte delle modifiche al testo da parte del redattore
    \item Minor Y: l'aumento della 'Y' indica una modifica testuale che è stata approvata dal verificatore
    \item Major X: l'incremento della 'X' si ha se è stata pubblicata dal responsabile una modifica non compatibile con le versioni precedenti del documento
\end{itemize}


\subsection{Verifica}
Durante questo processo, il compito dei verificatori è quello di effettuare l’analisi dei prodotti del team.
Tale analisi si differenzia in due diverse tipologie:
\begin{itemize}
    \item \textbf{Analisi statica}:processo di valutazione di un sistema o di un suo componente basato sulla sua
          forma, struttura, contenuto, documentazione. Questo tipo di analisi viene generalmente svolto
          tramite ispezioni e revisioni e può essere svolta sui documenti così come sul software o su parti
          di esso;

    \item \textbf{Analisi dinamica}: processo di valutazione di un sistema software o di un suo componente
          basato sull’osservazione del suo comportamento in esecuzione. Questo tipo di analisi viene
          generalmente chiamato testing e per motivi logici non può essere svolta sui documenti.
\end{itemize}


\subsubsection{Verifica della documentazione}
Nel caso la verifica venga fatta manualmente vengono usate le seguenti tecniche:
\begin{itemize}
    \item \textbf{Walkthrough}: Un controllo esaustivo del documento (è un processo oneroso che può richiedere molto tempo e risorse)
    \item \textbf{Inspection}: Un controllo parziale legato solo ai punti critici del documento
\end{itemize}
Il team ha deciso di utilizzare una tecnica \textbf{Walkthrough} nella fase iniziale del documento e prima del rilascio di versioni "major".
Dopo la formazione del documento ed un primo controllo esaustivo, si procede passando ad una tecnica di tipo \textbf{Inspection}
\paragraph{Strumenti utilizzati}
\begin{itemize}
    \item \url{https://www.italiancorrector.com/}
\end{itemize}
\subsubsection{Verifica del codice}

\subsection{Validazione}
Processo per assicurarsi che il prodotto sviluppato rispetti i requisiti concordati con il proponente.
Per fare ciò il prodotto verrà esaminato dai verificatori, verificando che esso sia conforme a quanto scritto nell'analisi dei requisiti. Sarà poi il responsabile successivamente che deciderà se accettare il prodotto o rifiutarlo chiedendo un ulteriore verifica.
\subsection{Comandi utili di Git}
I comandi più frequentemente utilizzati sono riassunti in seguito:
\begin{itemize}
    \item Sincronizzazione con repository remota: git pull;
    \item Creazione di un nuovo branch: git branch nome\_branch;
    \item Passaggio a un branch specificato: git checkout nome\_branch;
    \item Aggiunta delle modifiche alla stage area: git add file\_da\_aggiungere;
    \item Creazione del commit con le modifiche: git commit;
    \item Push sul remote di un nuovo branch: git push --set-upstream origin nome branch;
    \item Push su un branch già esistente: git push
\end{itemize}


\section{Processi organizzativi}
\subsection{Gestione dei processi}
\subsubsection{Organizzazione dei File}
Tutta la documentazione inerente al progetto sarà inclusa nella repository "Documentazione" all'interno dell'organizzazione chiamata "SwellFish".
I documenti di interesse per il proponente/committente si trovano in Documentazione/esterni, mentre quelli necessari al team di sviluppo si trovano in Documentazione/interni.
I verbali infine sono contenuti in Documentazione/esterni/verbali.

\subsubsection{Scrum}
Il team utilizzerà questo framework per gestire le varie fasi di sviluppo del progetto.
\paragraph{Organizzazione degli Sprint}
Ogni sprint avrà le seguenti caratteristiche:
\begin{itemize}
    \item Durata: una settimana. Inizia il venerdì e termina il venerdì successivo con la riunione concordata
    \item Riunione: viene effettuata il venerdì al termine dello sprint e vengono discussi i seguenti punti:
          \begin{itemize}
              \item Resoconto delle attività completate
              \item Problemi riscontrati e soluzioni adottate
              \item Analisi e resoconto delle ore e dei costi sostenuti
              \item Pianificazione data inizio prossimo sprint e obiettivi da fissare
          \end{itemize}
\end{itemize}
Gli obiettivi pianificati per lo sprint successivo verranno tradotti in "item" da aggiungere alla/e board presenti su Github e ad ogni item verranno assegnati dei membri del gruppo, coerentemente con la suddivisione dei ruoli per lo sprint in questione.

\subsubsection{Rotazione dei ruoli}
Vista l'adozione del framework scrum, il team ha deciso di far ruotare i ruoli dei componenti in corrispondenza della fine dello sprint, seguendo l'ordine alfabetico del nome dei componenti.
Pertanto ogni membro del team avrà impegni e responsabilità diverse ogni settimana. Questo approccio permette di ottenere un maggiore coinvolgimento.

\subsection{Gestione delle infrastrutture}
\subsubsection{Comunicazioni interne}
Le comunicazioni interne avvengono principalmente tramite:
\begin{itemize}
    \item \textbf{Discord}: Per quanto riguarda le riunioni settimanali e non. E' stato scelto in quanto conosciuto da tutti membri del gruppo, e la piattaforma web lo rende di facile accesso.
    \item \textbf{Telegram}: Per quanto riguarda comunicazioni minori o per comunicazioni di carattere logistico. Scelto in quanto conosciuto e usato da tutti i membri del gruppo.
\end{itemize}

\subsubsection{Comunicazioni esterne}
Le comunicazioni esterne avvengono principalmente tramite:
\begin{itemize}
    \item \textbf{E-mail} utilizzando l'email del gruppo "swellfish14@gmail.com"
    \item \textbf{Google meet} tramite link inviato dal proponente
    \item \textbf{Telegram} tramite apposito gruppo messo a disposizione dal proponente
\end{itemize}

\subsubsection{Gestione dei task}
Per organizzare meglio il lavoro e avere una panoramica chiara dello svolgimento delle attività e dello stato di avanzamento, è stato concordato l'uso della \textbf{Project Board} fornita da GitHub.
Le task riguardanti la documentazione e lo svolgimento del progetto vengo assegnate ai componenti dall'amministratore ed egli si occupa inoltre di chiuderle o modificarle in base alle esigenze.
Per agevolare l'attività di documentazione è stata creata una board denominata "To-Do Documenti", dove in concomitanza dello sprint settimanale vengono aggiunte nuove issue riguardanti la creazione o la modifica dei documenti da affrontare nel prossimo sprint.
Per gestire le attività di sviluppo verrà utilizzata un ulteriore board chiamata "SwellFish", al cui interno verranno inserite le task da completare in un dato sprint.
Queste board sono raggiungibili tramite l'apposita sezione "Projects" all'interno dell'organizzazione SwellFish.


\subsection{Miglioramento del processo}
Durante le riunioni settimanali, si guarda in retrospettiva il lavoro fatto nel periodo corrispondente e si valutano eventuali correzioni laddove si sono riscontrati problemi.
\end{document}