\documentclass[12pt]{article}
\usepackage{graphicx} % Required for inserting images
\usepackage{hyperref}
\usepackage{makecell}
\usepackage{makeidx}
\graphicspath{ {images/} }
\begin{document}

\title{SwellFish}
\begin{figure}
\centering
\includegraphics[scale=0.5]{Swellfish_logo.png}
\end{figure}
\author{Andrea Veronese, Claudio Giaretta, Elena Marchioro,\\
Davide Porporati, Francesco Naletto, Jude Vensil Barceros \\ \\
 \href{swellfish14@gmail.com}{swellfish14@gmail.com} \\
} 
\date{Marzo 2023}



\maketitle
\begin{center}
    \begin{tabular}{r | l}
		\multicolumn{2}{c}{\textit{Informazioni}}\\
		\hline
		
			\textit{Redattori} &
			[Andrea Veronese]\makecell{}\\
		
			\textit{Revisori} &
			[l]\makecell{}\\
			\textit{Responsabili} &
			[l]\makecell{}\\
		      \textit{Uso} & 
                [l]\makecell{}\\
\end{tabular}
\end{center}


\tableofcontents
\printindex 
\section{Introduzione}
Lo scopo del documento è quello di fornire un insieme di regole, strumenti e stadard di qualità necessari per creare un Way of Working condiviso e efficacemente impiegabile da tutti i membri del gruppo.
Questo documento conterrà tutte le comvenzioni da utilizzare per lo sviluppo del progetto e verrà aggiornato in maniera incrementale durante tutte le fasi dello sviluppo, pertanto è da considerarsi come work-in-progress.


\section{Modalità di sviluppo}
Data la necessità di efettuare un lavoro sostanzioso e condiviso, come tutti gli sviluppatori software, il team ha deciso di utilizzare una repository pubblica su GitHub per rendere agevole la collaborazione e la visione delle modifiche apportate alle varie parti del progetto.
E' stata quindi creata un'organizzazione chiamata SWEllFish e al suo interno sono state create delle specifiche repository:

\begin{itemize}
    \item SWELLFish: questa repo conterrà tutto il codice necessario per il progetto e i vari branch necessari per la suddivisione dei compiti
    \item Documentazione: questa repo contiene tutti i documenti utilizzabili sia dagli sviluppatori che dai committenti per avere una visione chiara delle modifiche apportate al progetto.
\end{itemize}

\section{Contatti con il committente/proponente}
Per semplificare la gestione dei contatti e delle eventuali richieste con l'azienda committente e con il Professore Vardenega, è stata creata una mail, \href{swellfish14@gmail.com}{} utilizzabile da tutti i componenti del gruppo.
L'uso di questa casella è stato regolamentato con un insieme di regole condivise dagli sviluppatori, pertanto un messaggio nasce dalla necessità di uno o più componenti del gruppo di avere delucidazioni da parte del committente o per sottoporre una modifica in attesa di accettazione.
I colloqui vengono richiesti via mail oppure via telegram, tramite l'apposito gruppoo creato con il responsabile incaricato dell'azienda proponente.

\section{Sviluppo}
La scadenza per la presentazione delle candidature dei capitolati è stata fissata per il 27/03/2023, pertanto il gruppo prevede di iniziare i lavori di analisi/sviluppo indicativamente il 03/04/2023, lasciando una settimana di "buco" per organizzare al meglio i compiti e le tempistiche.
\end{document}
