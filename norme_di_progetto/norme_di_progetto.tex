\documentclass[12pt]{article}
\usepackage{graphicx} % Required for inserting images
\usepackage{makeidx}
\usepackage{hyperref}
\usepackage{makecell}
\begin{document}


\graphicspath{ {../templates/img/} }
\graphicspath{ {./img/} }

\newcommand{\firstPage}{
    \begin{figure}
    \centering
    \includegraphics[scale=0.5]{Swellfish_logo.png}
    \end{figure}
    \author{Andrea Veronese, Claudio Giaretta, Elena Marchioro,\\
    Davide Porporati, Francesco Naletto, Jude Vensil Barceros \\ \\
    \href{swellfish14@gmail.com}{} \\
    } 
}

\title{Valutazione Capitolati}

\firstPage
\maketitle


\begin{center}
    \begin{tabular}{r | l}
		\multicolumn{2}{c}{\textit{Informazioni}}\\
		\hline
		
			\textit{Redattori} &
			[Andrea Veronese]\makecell{}\\
		
			\textit{Revisori} &
			[Davide Porporati]\makecell{}\\
			\textit{Responsabili} &
			[Francesco Naletto]\makecell{}\\
		      \textit{Uso} & 
                [Esterno]\makecell{}\\
\end{tabular}
\end{center}


\tableofcontents
\printindex 
\section{Introduzione}
Lo scopo del documento è quello di fornire un insieme di regole, strumenti e stadard di qualità necessari per creare un Way of Working condiviso e efficacemente impiegabile da tutti i membri del gruppo.
Questo documento conterrà tutte le comvenzioni da utilizzare per lo sviluppo del progetto e verrà aggiornato in maniera incrementale durante tutte le fasi dello sviluppo, pertanto è da considerarsi come work-in-progress.


\section{Processi Primari}
\subsection{Sviluppo}
Data la necessità di effettuare un lavoro sostanzioso e condiviso, come tutti gli sviluppatori software, il team ha deciso di utilizzare una repository pubblica su GitHub per rendere agevole la collaborazione e la visione delle modifiche apportate alle varie parti del progetto.
E' stata quindi creata un'organizzazione chiamata SWEllFish e al suo interno sono state create delle specifiche repository:

\begin{itemize}
    \item SWELLFish: questa repo conterrà tutto il codice necessario per il progetto e i vari branch necessari per la suddivisione dei compiti
    \item Documentazione: questa repo contiene tutti i documenti utilizzabili sia dagli sviluppatori che dai committenti per avere una visione chiara delle modifiche apportate al progetto.
    \item dev: repo utilizzata dal team per scrivere le bozze dei documenti
\end{itemize}
\section{Ruoli}
Vista l'adozione del framework scrum il gruppo ha deciso di far ruotare i ruoli dei vari componenti del gruppo in corrispondenza della fine dello sprint, seguendo un ordine di tipo alfabetico.
\section{Scrum}
Il team ha deciso di utilizzare questo framework per gestire le varie fasi di sviluppo del progetto.
\subsection{Organizzazione degli Sprint}
Ogni sprint avrà le seguenti caratteristiche:
\begin{itemize}
    \item Durata: una settimana. Inizia quindi il venerdì della settimana precedente e termina il venerdì successivo con la riunione concordata
    \item Riunione: viene effettuata il venerdì della settimana stessa e vengono discussi i seguenti punti:
    \begin{itemize}
        \item Resoconto dello sprint
        \item Problemi riscontrati e soluzioni adottate
        \item Analisi e resoconto delle ore e dei costi sostenuti
        \item Pianificazione data inizio prossimo sprint e obiettivi da fissare
    \end{itemize}
    
\end{itemize}


\subsection{Contatti con il committente/proponente}
Per semplificare la gestione dei contatti e delle eventuali richieste con l'azienda committente e con il Professore Vardenega, è stata creata una mail, \href{swellfish14@gmail.com}{} utilizzabile da tutti i componenti del gruppo.
L'uso di questa casella è stato regolamentato con un insieme di regole condivise dagli sviluppatori, pertanto un messaggio nasce dalla necessità di uno o più componenti del gruppo di avere delucidazioni da parte del committente o per sottoporre una modifica in attesa di accettazione.
Il canale telegram verrà usato per chiedere informazioni in modo informale con il proponente oppure per richiedere un confronto veloce 
I colloqui vengono richiesti via mail oppure via telegram, tramite l'apposito gruppo creato con il responsabile incaricato dell'azienda proponente.

\section{Organizzazione dei File}
Tutta la documentazione inerente al progetto sarà incusa nella repository "Documentazione" all'interno dell'organizzazione chiamata "SwellFish".
I documenti di interesse per il proponente/committente si trovano in Documentazione/esterni, mentre quelli necessari al team di sviluppo si trovano in Documentazione/interni. I verbali infine sono contenuti in Documentazione/esterni/verbali.



\section{Processi di supporto}
\subsection{Organizzazione delle task}
Per organizzare meglio il lavoro e avere una panoramica chiara dello svolgimento delle attività e dello stato di avanzamento, il gruppo ha concordato l'uso della Project Board fornita da GitHub.
Le task vengo assegnate ai componenti dal responsabile e si occupa inoltre di chiuderle o modificarle in base alle esigenze.

\subsection{Documentazione}
Ogni documento che verrà redatto attraverserà diversi processi, ogniuno dei quali vedrà impegnati uno o più componenti del gruppo.

Fasi:
\begin{itemize}
    \item Pianificazione: il file viene ideato e discusso con i membri del team sulla base delle necessità che deve soddisfare
    \item Stesura: il testo e il contenuto del documento viene realizzato dal redattore
    \item Revisione: il revisore incaricato si occupa di rileggerlo e correggere eventuali errori e/o imprecisioni, controllando che siano state rispettate le convenzioni stabilite in questo documento.
    \item Approvazione: il responsabile designato controlla che il documento sia corretto e coerente.
\end{itemize}
\subsection{Struttura di un verbale}


Per facilitare la produzione di documentazione e ridurre i tempi necessari per la redazione, è stato ideato un template generico da utilizzare.
Questo documento ha una struttura ben specifica:
\begin{itemize}
    \item Nome del gruppo
    \item Logo 
    \item Componenti del gruppo
    \item Data/Periodo di realizzazione
    \item Tabella con ruoli dei componenti del gruppo e finalità d'uso del documento
    \item Indice
\end{itemize}
\subsection{Struttura di un documento}
La struttura di un verbale utilizza la struttura di base del template impiegato per la stesura di un documento, con le seguenti aggiunte
\begin{itemize}
    \item Data
    \item Ora
    \item Durata
    \item Partecipanti
    \item Ordine del giorno
    \item Riassunto contenuti
\end{itemize}
\section{Versionamento}
Il VCS scelto dal gruppo per facilitare la gestione delle modifiche è Git, mediante l'utilizzo di GitHub.
A questo scopo è stata creata un'organizzazione denominata "SwellFish", al cui interno sono state create le opportune repository in base alle necessità del progetto.
Le varie repository presenti nell'organizzazione sono raggiungibili al seguente link : https://github.com/orgs/SWEllfish14/repositories.
\subsection{Regole per il versionamento}
Come prassi per la nomenclatura dei file il gruppo ha concordato la seguente struttura condivisa:
\begin{itemize}
    \item nome file: tutto in minuscolo, con le parole separate dall'underscore
    \item numero versione
    \item esempio: analisi_requisiti_v1.0.tex
\end{itemize}

Per la verifica e la validazione dei file il gruppo ha deciso di utilizzare la seguente prassi:
\begin{itemize}
    \item Creazione di un nuovo branch per un apposito documento
    \item Stesura del documento
    \item Caricamento sul branch appena creato
    \item Verifica della struttura, contenuti e rispetto delle convenzioni stabilite
    \item Pull request sul branch dev
    \item Rilascio sul branch master
\end{itemize}

Sono state inoltre utilizzate delle etichette e sono state create delle milestone che coincidono con le date di scadenza delle varie fasi del progetto
\begin{itemize}
    \item bug
    \item help

    
\end{itemize}
\subsection{Versionamento dei verbali}
Sono state applicate delle specifiche regole da adottare per la nomenclatura dei verbali e ogni documento dovrà rispettare la seguente struttura: verbale_data_azienda/nome_docente/attività svolta.
\subsection{Comandi utili di Git}
I comandi più frequentemente utilizzati sono riassunti in seguito:
\begin{itemize}
    \item Sincronizzazione con repository remota: git pull;
    \item Creazione di un nuovo branch: git branch nome branch;
    \item Passaggio a un branch specificata: git checkout nome branch;
    \item Aggiunta delle modifiche alla stage area: git add *;
    \item Creazione del commit con le modifiche: git commit;
    \item Push sul remote di un nuovo branch: git push --set-upstream origin nome branch;
    \item Push su un branch già esistente: git push
\end{itemize}

\end{document}
